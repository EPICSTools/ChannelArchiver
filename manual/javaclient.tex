\section{Java Archive Client} \label{sec:javaclient}
This tool is meant to be the main data retrieval tool. It provides a
graphical user interface to allow data browing. It is based on Java
and hence usable on many operating systems. It accesses the data via
the DataServer described in section \ref{sec:dataserver}, which means
that it can access the data via the network.  You invoke the java
archive client with the URL of the web server that hosts the archive
data server. The interactive GUI shown in Fig. \ref{fig:javatool} then
allows you to select one of the archives served by the network data
server, investigate the available channels etc.

In short, it's the greatest thing since canned beer. Once it's done.
\medskip

\begin{figure}[htb]
\begin{center}
\InsertImage{width=1.0\textwidth}{javaclient}
\end{center}
\caption{\label{fig:javatool}Java Archive Cient.}
\end{figure}