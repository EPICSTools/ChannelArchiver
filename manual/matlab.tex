\section{Matlab}
Programs like Matlab or Octave are ideally suited for the more
sophisticated analysis of archived data.  The ChannelArchiver
includes interface code for Matlab and Octave, allowing those two
programs to access data from the ChannelArchiver's Network Data
Server.  Refer to the file ChannelArchiver/Matlab/README for details
on building, installing and using those extensions.

\NOTE The Matlab/Octave support is experimental.
Its use is discouraged except for testing.

Figures \ref{fig:matlabtest}, \ref{fig:oildemo1},
\ref{fig:oildemo2}, \ref{fig:oildemo3} and
\ref{fig:oildemo4} showcase some examples and provide you with an
excuse to print at least this section of the manual on a color printer.

\begin{figure}[htb]
\begin{center}
\InsertImage{width=\textwidth}{matlabtest}
\end{center}
\caption{\label{fig:matlabtest}Matlab Example: Input and Output power
  of a Klystron for a two-day test run, combined with a scatter plot
  of the computed Klystron Gain.}
\end{figure}

\begin{figure}[htb]
\begin{center}
\InsertImage{width=\textwidth}{oildemo1}
\end{center}
\caption{\label{fig:oildemo1}Matlab Example: One-month overview of
  Klystron oil tank temperatures. The Plot-Binning request method as
  described in section \ref{sec:plotbinning} was used to reduce the
  amount of data. Interesting features like the noise on the DTL5
  signal as well as occasional spikes (which probably result from
  maintenance work on the oil tank) are preserved.
}
\end{figure}

\begin{figure}[htb]
\begin{center}
\InsertImage{width=\textwidth}{oildemo2}
\end{center}
\caption{\label{fig:oildemo2}Matlab Example: One-month overview of
  Klystron oil tank temperatures. The raw data was reduced by
  averaging into 1000 bins as described in section
  \ref{sec:lininterpol}, allowing for easier post-processing,
  and temperatures were converted to Celsius.
  Comparison with fig.\ \ref{fig:oildemo1} shows how many details can
  be lost via averaging, so it must be applied with caution.
}
\end{figure}

\begin{figure}[htb]
\begin{center}
\InsertImage{width=\textwidth}{oildemo3}
\end{center}
\caption{\label{fig:oildemo3}Matlab Example: The same data as
in \ref{fig:oildemo2} displayed as a ``Waterfall'' plot.
This type of display hides almost all the detail of the individual
channels. Outliers, however, stand out like the possible sensor problem
on DTL4, which is why this display method is well suited for an initial
investigation of many channels. The example also shows gaps in the data
caused by the many times when the archive engine was stopped during the
ongoing tests of the new archive engine.
}
\end{figure}

\begin{figure}[htb]
\begin{center}
\InsertImage{width=\textwidth}{oildemo4}
\end{center}
\caption{\label{fig:oildemo4}Matlab Example: Again the same data as
in \ref{fig:oildemo2} displayed as a surface plot.
Tools like Matlab allow the user to rotate this plot in real-time,
which might be useful for the inspection of certain channels.
}
\end{figure}
