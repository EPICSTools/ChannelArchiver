% waveforms.tex
\section{Waveform Data} \label{sec:waveforms}
In principle, the archiver toolset can be used for waveform channels
as well as scalar channels.
The archive engine stores the waveform data as received by ChannelAccess,
and you can get that information back out.
In practice, you might not be happy with the result because of basic
waveform related shortcomings in EPICS.

In all the cases listed below, the basic waveform samples and timestamp
provided by ChannelAccess and thus the archiver are incomplete.
One cannot display or otherwise use the waveform data without
additional meta information like sample rate, positions etc.
Some sites develop custom ChannelAccess clients which obtain this
information from user input, a relational database,
or other channels, maybe determining the latter based on a naming standard.
The generic archiver toolkit only provides the basic data, no
\INDEX{waveform meta information}.

\begin{itemize}
\item Fast-ADC-Type Data\\
Your waveform data might represent the signal of a fast ADC.
ChannelAccess and thus the archiver toolkit present you with
the waveform samples, which probably stand for the voltages
sampled by the ADC. You also get one timestamp. It might indicate
the time when the ADC was triggered, which might be the time stamp
of the first sample in the waveform.
To use the data, you also need to know the sample rate,
that is the time period between samples, but neither ChannelAccess
nor the archiver hand you that number.

\item Multi Channel Analyzer or Scanner Type Data\\
In this case, the waveform samples might represent the intensities
or histogram count of something that was measured at different
positions or at different energies.
To use that data, you also need that position or energy or other
characteristic information for each sample.

\item Images\\
Some waveforms are used to represent images, where for example
the first 640 samples of a waveform represent the first "line"
of the image, the next 640 samples the next line and so on.
Nothing tells you the line width, if one sample represents
gray-scale intensity, or if 3 samples together represent red/green/blue
etc.

\item Custom Structures\\
In other cases, EPICS waveforms are used to exchange custom data,
ranging from strings that exceed 40 characters to arbitrary
binary data. You need to know how to decode the information.
\end{itemize}
