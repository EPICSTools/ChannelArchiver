\chapter{ArchiveEngine}

\section{Configuration}
To be in XML.

\section{Starting and Stopping}
As before.

\section{Web Interface}
As before.

\section{Threads}
The ArchiveEngine uses several threads:
\begin{itemize}
\item A main thread that reads the initial configuration and then
  enters a main loop for the periodic scan lists and writes to the
  disk.
\item The ChannelAccess client library is used in its multi-threaded
  version. The internals of this are beyond the control of the
  ArchiveEngine, the total number of CA client threads is unknown.
\item The ArchiveEngine's HTTP (web) server runs in a separate thread,
  with each HTTP client connection again being handled by its own
  thread. The total number of threads therefore depends on the number
  of current web clients.
\end{itemize}
As a result, the total number of threads changes at runtime. Though
these internals should not be of interest to end users, this can be
confusing especially on older releases of Linux where each thread
shows up as a process in the process list.
On Linux version 2.2.17-8 for example we get process table entries as
shown in Tab.~\ref{lst:aeprocs} for a single ArchiveEngine, connected
to four channels served by excas, no current web client. The only hint
we get that this is in fact one and the same ArchiveEngine lies in the
consecutive process IDs.

\begin{lstlisting}[float=htb,
caption={Output of Linux 'ps' process list command, see text.},
label=lst:aeprocs]
  PID TTY          TIME CMD
29721 pts/5    00:00:00 ArchiveEngine
29722 pts/5    00:00:00 ArchiveEngine
29723 pts/5    00:00:00 ArchiveEngine
29724 pts/5    00:00:00 ArchiveEngine
29725 pts/5    00:00:00 ArchiveEngine
29726 pts/5    00:00:00 ArchiveEngine
29727 pts/5    00:00:00 ArchiveEngine
29728 pts/5    00:00:00 ArchiveEngine   
\end{lstlisting}

The first conclusion is that one should not be surprised to see
multiple ArchiveEngine entries in the process table.
The other issue arises when one tries to 'kill' a running
ArchiveEngine. Though the preferred method is via the engine's web
interface, one can try to send a signal to the first process, the one
with the lowest PID.

