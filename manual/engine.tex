\chapter{ArchiveEngine}

\begin{figure}[htb]
\begin{center}
\InsertImage{width=\textwidth}{engine}
\end{center}
\caption{\label{fig:engine}Archive Engine, refer to text.}
\end{figure}

\noindent The ArchiveEngine is an EPICS ChannelAccess client. It can
save any channel served by any ChannelAccess server. One ArchiveEngine
can archive data from more than one CA server. For more details on the
CA server data sources, refer to section \ref{sec:datasource} on page
\pageref{sec:datasource}.  The ArchiveEngine supports the sampling
options that were described in section \ref{sec:sampling} on page
\pageref{sec:sampling}.  The ArchiveEngine is configured with an XML
file that lists what channels to archive and how. Each given channel
can have a different periodic scan rate or be archived in monitor mode
(on change).  One design target was: Archive 10000 values per second,
be it 1000 channels that change at 10Hz each or 10000 channels which
change at 1Hz.

The ArchiveEngine saves the full information available via
ChannelAccess: The value, time stamp and status as well as
control information like units, display and alarm limits, ...  
The data is written to an archive in the form of local disk files,
specifically index and data files.  Chapter \ref{chap:storage}
provides details on the file formats.
While running, status and configuration of the ArchiveEngine are
accessible via a built-in web server, accessible via any web browser
on the network.  The chapter on data retrieval, beginning on page
\pageref{chap:retrieval}, introduces the available retrieval tools
that allow users to look at the archived data.

\section{Configuration}
The ArchiveEngine expects an XML-type configuration file that follows
the document type description format from listing
\ref{lst:engineconfigdtd}. Listing \ref{lst:engineconfigex} provides
an example. In the following subsections, we describe the various XML
elements of the configuration file.

\lstinputlisting[float=htb,keywordstyle=\sffamily,caption={XML DTD for
    the Archive Engine
    Configuration},label=lst:engineconfigdtd]{../Engine/engineconfig.dtd}

\lstinputlisting[float=htb,keywordstyle=\sffamily,caption={Example
    Archive Engine Configuration},label=lst:engineconfigex]{../Engine/engineconfig.xml}

\clearpage

\subsection{write\_period}
This is a \INDEX{Global Option} that needs to precede any group and
channel definitions.  It configures the write period of the Archive
Engine in seconds. The default value of 30 seconds means that the
engine will write to Storage every 30 seconds.

\subsection{get\_threshold}
This global option determines when the archive engine switches from
``Sampled'' operation to ``Sampled using monitors'' as described in
section \ref{sec:sampling}.

\subsection{file\_size}
This global option determines when the archive engine will create a
new data file. The default of 100 means that the engine will continue
to write to a data file until that file reaches a size of
approximately 100~MB, at which point a new data file is created.

\subsection{ignored\_future}
Defines ``too far in the future'' as ``now $+$ ignored\_future''.
Samples with time stamps beyond that time are ignored.  Details: For
strange reasons, the Engine sometimes receives values with invalid
time stamps. The most common example is a ``Zero'' time stamp: After
an IOC reboots, all records have a zero time stamp until they are
processed. For passive records, as commonly used for operator input,
this time stamp will stay zero until someone enters a value on an
operator screen or via a safe/restore utility. The Engine cannot
archive those values because the retrieval relies on the values being
sorted in time. A zero time stamp does not fit in.

Should an IOC (for some unknown reason) produce a value with an
outrageous time stamp, e.g. "1/2/2035", another problem occurs: Since
the archiver cannot go back in time, it cannot add further values to
this channel until the date "1/2/2035" is reached.  Consequently,
future time stamps have to be ignored. (default: 6h)

\subsection{buffer\_reserve} \label{sec:reserve}
To buffer the samples between writes to the disk, the engine keeps a
memory buffer for each channel. The size of this buffer is
$$ buffer\_reserve \times \frac{write\_period}{scan\_period}
$$ Since writes can be delayed by other tasks running on the same
computer as well as disk activity etc., the buffer is bigger than the
minimum required: buffer\_reserve defaults to 3.

\subsection{group}
Every channel belongs to a group of channels. The configuration file
must define at least one group. For organizational or aesthetic
purposes, you might add more groups. One important use of groups is
related to the ``disable'' feature, see section \ref{sec:disable}.

\subsubsection{name}
This mandatory sub-element of a group defines its name.

\subsection{channel}
This element defines a channel by providing its name and the sampling
options. A channel can be part of more than one group. To accomplish
this, simply list the channel as part of all the groups to which it
should belong.

\subsubsection{name}
This mandatory sub-element of a channel defines its name. Any name
acceptable for ChannelAccess is allowed. The archive engine does not
perform any name checking, it simply passes the name on to the CA
client library, which in turn tries to resolve the name on the
network.  Ultimately, the configuration of your data servers decides
what channel names are available.

\subsubsection{period}
This mandatory sub-element of a channel defines the sampling period.
In case of periodic sampling, this is the period at which the periodic
sampling attempts to operate. In case of monitored channels (see next
option), this is the estimated rate of change of the channel.
The period is specified in units of seconds.

If a channel is listed more than once, for example as part of
different groups, the channel will still only be sampled once. The
sampling mechanism is determined by maximizing the data rate. If, for
example, the channel ``X'' is once configured for periodic sampling
every 30 seconds and once as a monitor with an estimated period or one
second, the channel will in fact be monitored with an estimated period
of 1 second.

\subsubsection{monitor}
Either ``monitor'' or ``scan'' needs to be provided as part of a
channel configuration to select the sampling method.  In the case of
``monitor'', the channel will be monitored, that is: Each change
received via ChannelAccess will be stored. The ``period'' tag is used
to determine the in-memory buffer size of the engine. That means: If
samples arrive much more frequently than estimated via the ``period''
tag, the archive engine might drop samples. (See also
``buffer\_reserve'', \ref{sec:reserve}

\subsubsection{scan}
As an alternative to the ``monitor'' tag, ``scan'' can be used to
select periodic sampling. The preceding ``period'' tag determines the
sampling period.

\subsubsection{disable} \label{sec:disable}
This optional sub-element of a channel turns the channel into a
``disabling'' channel for the group. Whenever the value of the channel
is above zero, sampling of the whole group will be disabled until the
channel returns to zero or below zero.

This is useful for e.g.\ a group of channels related to power
supplies: Whenever the power supply is off, we might want to disable
scanning of the power supplies' voltage and current because those
channels will only yield noise. By disabling the sampling based on a
``Power Supply is Off'' channel, we can avoid storing those values
which are of no interest.

\NOTE There is no ``enabling'' feature (yet), meaning: The channel
marked as ``disable'' will disable its group whenever it is above
zero. There is no ``enable'' flag that would enable archiving of a group
whenever the flagged channel is above zero.

\section{Starting and Stopping}
\subsection{Starting}
The ArchiveEngine is a command-line program that displays usage
information similar to the following:

\begin{lstlisting}[frame=none,keywordstyle=\sffamily]
ArchiveEngine Version 2.1.0, EPICS 3.14.4
 
USAGE: ArchiveEngine [Options] <config-file> <index-file>
 
Options:
  -port <port>         Web server TCP port
  -description <text>  description for HTTP display
  -log <filename>      write logfile
  -nocfg               disable online configuration
\end{lstlisting}

\noindent Minimally, the engine is therefore started by simply naming the
configuration file and the path to the index file, which can be in the
local directory:

\begin{lstlisting}[frame=none,keywordstyle=\sffamily]
ArchiveEngine engineconfig.xml ./index
\end{lstlisting}

\noindent After collecting some data, the ArchiveEngine will create the
specified index file together with data files in the same directory
that contains the index file.

\subsection{``-log'' Option}
This option causes the ArchiveEngine to create a log file into which
all the messages that otherwise only appear on the standard output are copied.

\subsection{``-description'' Option}
This option allows setting the description string that gets displayed
on the main page of the engine's built-in HTTP server, see
section \ref{sec:enginehttp}.

\subsection{``-port'' Option}
This option configures the TCP port of the engine's HTTP server, again
see section \ref{sec:enginehttp}. The default port number is 4812.

If you think this number stinks for a default, you are not too far off
base: In Germany, there is a very well known Au-de-Cologne called
4711.  Since forty-seven-eleven is therefore easily remembered by
anybody from Germany, adding 1 to each 47 and 11 naturally results in
an equally easy to remember 4812.
And for those who fail to appreciate the German-centered default port
number, the ``-port'' option allows you to pick a number of your
personal fancy.

\subsection{``-nocfg'' Option}
This option disables the ``Config'' page of the engine's HTTP server,
in case you want to prohibit online changes.

\subsection{The ``archive\_active.lck'' File}
You can only run one ArchiveEngine per directory because it creates
the index and data files in there.  When running, this lock file is
created. The ArchiveEngine will refuse to run if this file already
exists.  After shutdown, the ArchiveEngine will remove this lock file.
If the ArchiveEngine crashes or is not stopped gracefully by the
operating system, this lock file will be left behind.  You cannot
start the ArchiveEngine again until you remove the lock file. This is
a reminder for you to check the cause of the improper shutdown and
maybe check the data files for corruption.

\NOTE This is no 100\%\ dependable check. Data corruption occurs when
two engines attempt to write to the same index and data files. The
lock file, however, is created in the directory where the
ArchiveEngine was started, which could be different from the directory
where the data gets written. Example:

\begin{lstlisting}[frame=none,keywordstyle=\sffamily]
cd /some/dir
ArchiveEngine -p 7654 engineconfig.xml /my/data/index &

cd /another/dir
ArchiveEngine -p 7655 engineconfig.xml /my/data/index &
\end{lstlisting}

\noindent This is a sure-fire way to corrupt the data in
``/my/data/index'' and the accompanying data files because two
ArchiveEngines are writing to the same archive.

\subsection{More than one ArchiveEngine}
You can run multiple ArchiveEngines on the same
computer. But they must
\begin{enumerate}
\item be in separate directories. See the preceding discussion of
      the lock file which is meant to assist in avoiding this
      problem.
\item use a different TCP port number for the built-in web server
\end{enumerate}
In practice this means that you have to create different directories
on the disk, one per ArchiveEngine, and in there run the
ArchiveEngines with different "-p $<$port$>$" options.

\subsection{Stopping}
While the ArchiveEngine can be stopped by pressing ``CTRL-C''
or using the equivalent ``kill'' command in Unix, the preferred method is
via the built-in web server. Use any web browser and point it to

\begin{lstlisting}[frame=none,keywordstyle=\sffamily]
http://<host where engine is running>:<port>/stop
\end{lstlisting}

\noindent Per default, the engine uses 4812, so you could use the
following URL to stop that engine on the local computer:

\begin{lstlisting}[frame=none,keywordstyle=\sffamily]
http://localhost:4812
\end{lstlisting}

% -------------------------------------------------------------------------------
\section{Web Interface} \label{sec:enginehttp} % --------------------------------
\begin{figure}[htb]
\begin{center}
\InsertImage{width=0.8\textwidth}{engine_main}
\end{center}
\caption{\label{fig:engine:main}Main Page of Archive Engine's HTTPD}
\end{figure}

The ArchiveEngine has a built-in web server (HTTP Daemon) for status
and configuration information.  You can use any web browser to access
this web server.  You can do that on the computer where the
ArchiveEngine is running as well as from other computers, be it a PC
or Macintosh or other system as long as that computer can reach the
machine that is running the ArchiveEngine via the network.  You do
\emph{not} need a web server like the Apache web server for Unix or
the Internet Information Server for Win32 to use this. The
ArchiveEngine \emph{itself} acts as a web server.

You \emph{cannot} view archived data with this mechanism.  See the
documentation on data retrieval (chapter \ref{chap:retrieval}) for
that, because the archive engine's HTTPD is meant for access to the
status and configuration of the running engine, not for accessing the
data samples.

To access the ArchiveEngine's web server, you need to know the
Internet name of the machine that is running the ArchiveEngine as well
as the TCP port.  If you are on the same machine, use ``localhost''.
The port is configured when you start the ArchiveEngine, it defaults
to 4812. Then use any web browser and point it to

\begin{lstlisting}[frame=none,keywordstyle=\sffamily]
 http://<host where engine is running>:<port>
\end{lstlisting}

\noindent Example for an ArchiveEngine running on the local machine with the
default port number:

\begin{lstlisting}[frame=none,keywordstyle=\sffamily]
http://localhost:4812
\end{lstlisting}

\noindent The start page of the ArchiveEngine web server should look
similar to the one shown in Fig.~\ref{fig:engine:main}.


\section{Threads} % ----------------------------------------------------------
The ArchiveEngine uses several threads:
\begin{itemize}
\item A main thread that reads the initial configuration and then
  enters a main loop for the periodic scan lists and writes to the
  disk.
\item The ChannelAccess client library is used in its multi-threaded
  version. The internals of this are beyond the control of the
  ArchiveEngine, the total number of CA client threads is unknown.
\item The ArchiveEngine's HTTP (web) server runs in a separate thread,
  with each HTTP client connection again being handled by its own
  thread. The total number of threads therefore depends on the number
  of current web clients.
\end{itemize}
As a result, the total number of threads changes at runtime. Though
these internals should not be of interest to end users, this can be
confusing especially on older releases of Linux where each thread
shows up as a process in the process list.
On Linux version 2.2.17-8 for example we get process table entries as
shown in Tab.~\ref{lst:aeprocs} for a single ArchiveEngine, connected
to four channels served by excas, no current web client. The only hint
we get that this is in fact one and the same ArchiveEngine lies in the
consecutive process IDs.

\begin{lstlisting}[float=htb,
caption={Output of Linux 'ps' process list command, see text.},
label=lst:aeprocs]
  PID TTY          TIME CMD
29721 pts/5    00:00:00 ArchiveEngine
29722 pts/5    00:00:00 ArchiveEngine
29723 pts/5    00:00:00 ArchiveEngine
29724 pts/5    00:00:00 ArchiveEngine
29725 pts/5    00:00:00 ArchiveEngine
29726 pts/5    00:00:00 ArchiveEngine
29727 pts/5    00:00:00 ArchiveEngine
29728 pts/5    00:00:00 ArchiveEngine   
\end{lstlisting}

The first conclusion is that one should not be surprised to see
multiple ArchiveEngine entries in the process table.
The other issue arises when one tries to 'kill' a running
ArchiveEngine. Though the preferred method is via the engine's web
interface, one can try to send a signal to the first process, the one
with the lowest PID.

