\chapter{\INDEX{ArchiveEngine}} \label{sec:engine}

\begin{figure}[htb]
\begin{center}
\InsertImage{width=\textwidth}{engine}
\end{center}
\caption{\label{fig:engine}Archive Engine, refer to text.}
\end{figure}

\noindent The ArchiveEngine is an EPICS ChannelAccess client. It can
save any channel served by any ChannelAccess server. One ArchiveEngine
can archive data from more than one CA server. For more details on the
CA server data sources, refer to section \ref{sec:datasource} on page
\pageref{sec:datasource}.  The ArchiveEngine supports the sampling
options that were described in section \ref{sec:sampling} on page
\pageref{sec:sampling}.  The ArchiveEngine is configured with an XML
file that lists what channels to archive and how. Each given channel
can have a different periodic scan rate or be archived in monitor mode
(on change).  One design target was: Archive 10000 values per second,
be it 1000 channels that change at 10Hz each or 10000 channels which
change at 1Hz.

The ArchiveEngine saves the full information available via
ChannelAccess: The value, time stamp and status as well as
control information like units, display and alarm limits, ...  
The data is written to an archive in the form of local disk files,
specifically index and data files.  Chapter \ref{chap:storage}
provides details on the file formats.
While running, status and configuration of the ArchiveEngine are
accessible via a built-in web server, accessible via any web browser
on the network.  The chapter on data retrieval, beginning on page
\pageref{chap:retrieval}, introduces the available retrieval tools
that allow users to look at the archived data.

\section{Configuration} \label{sec:engineconfig}
The ArchiveEngine expects an XML-type configuration file that follows
the document type description format from listing
\ref{lst:engineconfigdtd} (see section \ref{sec:dtdfiles} on DTD file
installation). Listing \ref{lst:engineconfigex} provides an
example. In the following subsections, we describe the various XML
elements of the configuration file.

\lstinputlisting[float=htb,keywordstyle=\sffamily,caption={XML DTD for
    the Archive Engine
    Configuration},label=lst:engineconfigdtd]{../Engine/engineconfig.dtd}

\lstinputlisting[float=htb,keywordstyle=\sffamily,caption={Example
    Archive Engine Configuration},label=lst:engineconfigex]{../Engine/engineconfig.xml}

\clearpage

\subsection{\INDEX{write\_period tag}}
This is a \INDEX{global option} that needs to precede any group and
channel definitions.  It configures the write period of the Archive
Engine in seconds. The default value of 30 seconds means that the
engine will write to Storage every 30 seconds.

\subsection{\INDEX{get\_threshold tag}} \label{sec:getthreshold}
This global option determines when the archive engine switches from
``Sampled'' operation to ``Sampled using monitors'' as described in
section \ref{sec:sampling}. Defaults to 20 seconds.

\subsection{\INDEX{file\_size tag}}
This global option determines when the archive engine will create a
new data file. The default of 100 means that the engine will continue
to write to a data file until that file reaches a size of
approximately 100~MB, at which point a new data file is created.

\subsection{\INDEX{ignored\_future tag}}
Defines ``too far in the future'' as ``now $+$ ignored\_future''. It
is specified in hours, and samples with time stamps beyond that time
are ignored.
Details: For strange reasons, the Engine sometimes receives values with invalid
time stamps. The most common example is a ``Zero'' time stamp: After
an IOC reboots, all records have a zero time stamp until they are
processed. For passive records, as commonly used for operator input,
this time stamp will stay zero until someone enters a value on an
operator screen or via a save/restore utility. The Engine cannot
archive those values because the retrieval relies on the values being
sorted in time. A zero time stamp does not fit in.

Should an IOC (for some unknown reason) produce a value with an
outrageous time stamp, e.g. "1/2/2035", another problem occurs: Since
the archiver cannot go back in time, it cannot add further values to
this channel until the date "1/2/2035" is reached.  Consequently,
future time stamps have to be ignored. (default: 6h)

\subsection{\INDEX{buffer\_reserve tag}} \label{sec:reserve}
To buffer the samples between writes to the disk, the engine keeps a
memory buffer for each channel. The size of this buffer is rounded up
to the next integer from
$$ buffer\_reserve \times \frac{write\_period}{scan\_period}
$$ Since writes can be delayed by other tasks running on the same
computer as well as disk activity etc., the buffer is bigger than the
minimum required: buffer\_reserve defaults to 3.

\subsection{\INDEX{max\_repeat\_count tag}} \label{sec:repeats}
When sampling in a scanned mode (as opposed to monitored), the engine
stores only new values. As long as a value matches the preceding
sample, it is not written to storage. Only after the value changes, a
special value marked with a severity of ARCH\_REPEAT and a status that
reflects the number of repeats is written, then the new sample is
added.

This procedure conserves disk space. The disadvantage lies in the fact
that one does not see any new samples in the archive until the
channel changes, which can be disconcerting to some users. Therefore
the max\_repeat\_count configuration parameter was added. It forces
the engine to write a sample even if the channel has not change after
the given number of repeats. The default is 120, meaning that a
channel that is scanned every 30 seconds will be written once an hour
even if it had not changed. 

\subsection{\INDEX{disconnect tag}} \label{sec:disconnect}
This global option selects how ``disabled'' channels (see
\ref{sec:disable}) are handled. By default, disabled channels will
stay connected via ChannelAccess, but no values are archived.
When setting the ``disconnect'' option, disabled channels will instead
disconnect from ChannelAccess and then, later, attempt to reconnect
once the channel is again enabled.

In general, it is a good idea to stay with the default. That way we
leave the connection handling to the ChannelAccess client library,
which is optimized to do this. The engine will still receive new data,
and as soon as the channel is re-enabled, it can thus store the most
recent value.

The disconnect feature was added for the rare case that you have IOCs
that are temporarily off-line, and some PV will tell you about the
fact. You can then use that PV to disable and disconnect the affected
channels, preventing the ChannelAccess client library from continuing
to issue connection attempts. Another example would be that you want
to reduce the network load of continuing CA monitors for channels
that are archived via monitors at a high rate but disabled. 
Most likely, though, checking your channels' update rates or using a
temporary archive engine might be the better solution.

\subsection{\INDEX{group tag}}
Every channel belongs to a group of channels. The configuration file
must define at least one group. For organizational or esthetic
purposes, you might add more groups. One important use of groups is
related to the ``disable'' feature, see section \ref{sec:disable}.

\subsubsection{\INDEX{name tag}}
This mandatory sub-element of a group defines its name.

\subsection{\INDEX{channel tag}}
This element defines a channel by providing its name and the sampling
options. A channel can be part of more than one group. To accomplish
this, simply list the channel as part of all the groups to which it
should belong.

\subsubsection{\INDEX{name tag}}
This mandatory sub-element of a channel defines its name. Any name
acceptable for ChannelAccess is allowed. The archive engine does not
perform any name checking, it simply passes the name on to the CA
client library, which in turn tries to resolve the name on the
network.  Ultimately, the configuration of your data servers decides
what channel names are available.

\subsubsection{\INDEX{period tag}} \label{sec:period}
This mandatory sub-element of a channel defines the sampling period.
In case of periodic sampling, this is the period at which the periodic
sampling attempts to operate. In case of monitored channels (see next
option), this is the estimated rate of change of the channel.
The period is specified in units of seconds.

If a channel is listed more than once, for example as part of
different groups, the channel will still only be sampled once. The
sampling mechanism is determined by maximizing the data rate. If, for
example, the channel ``X'' is once configured for periodic sampling
every 30 seconds and once as a monitor with an estimated period or one
second, the channel will in fact be monitored with an estimated period
of 1 second.

\subsubsection{\INDEX{scan tag}}
Either ``monitor'' or ``scan'' need to be provided as part of a
channel configuration to select the sampling method.  
True to its name, ``scan'' selects scanned operation, where the preceding
``period'' tag determines the sampling period, that is the time
between taking samples.
As an example, scanned operation with a period of 60 means: Every 60
seconds, the engine will write the most recent value of the channel to
the archive.

\subsubsection{\INDEX{monitor tag}}
As an alternative to the ``scan'' tag, ``monitor'' can be used,
requesting monitored operation, that is: An attempt is made to store
each change received via ChannelAccess. The ``period'' tag is used to
determine the in-memory buffer size of the engine. That means: If
samples arrive much more frequently than estimated via the ``period''
tag, the archive engine might drop samples. (See also
``buffer\_reserve'', \ref{sec:reserve}).


\subsubsection{\INDEX{disable tag}} \label{sec:disable}
This optional sub-element of a channel turns the channel into a
``disabling'' channel for the group. Whenever the value of the channel
is above zero, sampling of the whole group will be disabled until the
channel returns to zero or below zero
(see \ref{sec:disconnect} for additional disconnection).

This is useful for e.g.\ a group of channels related to power
supplies: Whenever the power supply is off, we might want to disable
scanning of the power supplies' voltage and current because those
channels will only yield noise. By disabling the sampling based on a
``Power Supply is Off'' channel, we can avoid storing those values
which are of no interest.

The channel which is ``disabling'' its group will stay enabled.
Internal to the archive engine, it obviously needs to stay connected
and enabled. How else would it otherwise learn when to re-enable
its group? Since it might be of interest to learn which values of the
``disabling'' channel caused the other channels in the group to be enabled
or disabled, its samples are also added to the archive: Disabling channels
are never disabled themselves.

\NOTE There is no ``enabling'' feature, meaning: The channel
marked as ``disable'' will disable its group whenever it is above
zero. There is no ``enable'' flag that would enable archiving of a group
whenever the flagged channel is above zero.
If you want it the other way around, you typically add a CALC record
to handle the inversion.

\section{Example for Sampling a Channel}
Assume that a channel ``fred'' emits monitors at 1~Hz.
These are some examples for sampling it, and what one can expect to
find in the archive as a result.
\begin{itemize}
\item ``fred 1 Monitor''\\
      Every value sent by fred is archived. Might be a good idea for
      some channels, but don't try to store every value of every PV
      of your control system indefinitely unless you are prepared to
      deal with that amount of data.

      Per default, the engine will write every 30 seconds. So it will
      have to allocate a buffer for about 30 samples, based on our
      estimate of 1 second between incoming monitors.
      With the default buffer\_reserve of 3, it will actually allocate
      a buffer for 90 values, so we don't loose data when the engine
      should get delayed in writing. On the other hand, when the
      computer is terribly busy, we might not receive any more values,
      either. In any case, the chances of overflowing the data buffer
      are slim.
\item ``fred 1''\\
      The engine will sample once per second, and the channel changes
      once a second, so you might think that you archive every value
      just as in the previous example.  Well, the sample period of the
      engine running on the host and the scanning of the channel on
      the CA server are not synchronized, plus there are additional
      network delays. So you will sometimes miss values whenever more
      than one sample arrived between the engine's sampling, or get
      duplicate values whenever no new value arrived between the
      engine's sampling. Bad idea.

      Except: With the default get\_threshold of 20 seconds, the engine
      will \emph{not} issue a 'get' every second. It will instead use a
      monitor, and ignore all data that arrives faster than one
      second. So this specific case will probably give the exact same result
      as the previous case!
\item ``fred 60''\\
      The engine will sample every 60 seconds. This is a very
      reasonable setup: The channel samples at 1 Hz, so you get
      frequent updates for the operator interface, but for the archive
      we only care about a sample per minute and save storage space by
      ignoring finer detail.

      With the default get\_threshold, that's it. If you raised the
      threshold, for example to 70 seconds, the engine would use
      monitors, so it would receive the 1 Hz data and ignore 59
      samples each minute. That is probably a waste of network
      bandwidth. You might, on the other hand, get more consistent
      time stamps, since the '60 second' period is now based on the
      time stamps which the IOC sends, and not the host clock.

      While this sounds like a neat trick, it might be cleaner to
      create a channel on the CA server which only updates every 60
      seconds, then use ``fred 60 Monitor'' to store each such sample.
\item ``fred 60 Monitor''\\
      Probably an error. The engine is instructed to save every
      incoming monitor. Expecting one value to arrive about every 60
      seconds, and assuming a write period of 30 seconds with buffer
      reserve of 3, the engine will allocate a buffer for 3 values.

      In reality, however, about 30 values arrive in every 30 second
      write cycle. You will see buffer ``overrun'' errors, because the
      engine overwrites older samples in its ring buffer with newly
      arriving samples, and the archive will contain the last 3
      samples that happened to be in the buffer at write-to-disk time.
\end{itemize}

\section{Starting and Stopping}
\subsection{Starting}
The ArchiveEngine is a command-line program that displays usage
information similar to the following:

\begin{lstlisting}[frame=none,keywordstyle=\sffamily]
USAGE: ArchiveEngine [Options] <config-file> <index-file>
 
Options:
  -port <port>         Web server TCP port
  -description <text>  description for HTTP display
  -log <filename>      write logfile
  -nocfg               disable online configuration
\end{lstlisting}

\noindent Minimally, the engine is therefore started by simply naming the
configuration file and the path to the index file, which can be in the
local directory:

\begin{lstlisting}[frame=none,keywordstyle=\sffamily]
ArchiveEngine engineconfig.xml ./index
\end{lstlisting}

\noindent After collecting some data, the ArchiveEngine will create the
specified index file together with data files in the same directory
that contains the index file.

\subsection{``-log'' Option}
This option causes the ArchiveEngine to create a log file into which
all the messages that otherwise only appear on the standard output are
copied.

This, however, only applies to messages which the engine writes
out. Other code, including the ChannelAccess client library, might
produce further messages, which still go the the standard output or
error output streams.

\subsection{``-description'' Option} \label{sec:enginedesc}
This option allows setting the description string that gets displayed
on the main page of the engine's built-in HTTP server, see
section \ref{sec:enginehttp}.

\subsection{``-port'' Option} \label{sec:engineport}
This option configures the TCP port of the engine's HTTP server, again
see section \ref{sec:enginehttp}. The default port number is 4812.

If you think this number stinks for a default, you are not too far off
base: In Germany, there is a very well known Au-de-Cologne called
4711.  Since forty-seven-eleven is therefore easily remembered by
anybody from Germany, adding 1 to each 47 and 11 naturally results in
an equally easy to remember 4812.
No, the archiver development is not funded by the 4711 company.

\subsection{``-nocfg'' Option}
This option disables the ``Config'' page of the engine's HTTP server,
in case you want to prohibit online changes.

\subsection{The ``archive\_active.lck'' File}
You can only run one ArchiveEngine per directory. This is meant to
prevent duplicate startups of the same engine, potentially damaging
the index and data files. When running, this lock file is
created. The ArchiveEngine will refuse to run if this file already
exists.  After shutdown, the ArchiveEngine will remove this lock file.
If the ArchiveEngine crashes or is not stopped gracefully by the
operating system, this lock file will be left behind.  You cannot
start the ArchiveEngine again until you remove the lock file. This is
a reminder for you to check the cause of the improper shutdown and
maybe check the data files for corruption.

\NOTE This is no 100\%\ dependable check. Data corruption occurs when
two engines attempt to write to the same index and data files. The
lock file, however, is created in the directory where the
ArchiveEngine was started, which could be different from the directory
where the data gets written. Example:

\begin{lstlisting}[frame=none,keywordstyle=\sffamily]
cd /some/dir
ArchiveEngine -p 7654 engineconfig.xml /my/data/index &

cd /another/dir
ArchiveEngine -p 7655 engineconfig.xml /my/data/index &
\end{lstlisting}

\noindent This is a sure-fire way to corrupt the data in
``/my/data/index'' and the accompanying data files because two
ArchiveEngines are writing to the same archive.

\subsection{More than one ArchiveEngine}
You can run multiple ArchiveEngines on the same
computer. But they must
\begin{enumerate}
\item be in separate directories, writing to different archives.
      See the preceding discussion of the lock file.
\item use a different TCP port number for the built-in web server
\end{enumerate}
In practice this means that you have to create different directories
on the disk, one per ArchiveEngine, and in there run the
ArchiveEngines with different "-p $<$port$>$" options.

\subsection{Stopping}
While the ArchiveEngine can be stopped by pressing ``CTRL-C''
or using the equivalent ``kill'' command in Unix, the preferred method is
via the built-in web server. Use any web browser and point it to

\begin{lstlisting}[frame=none,keywordstyle=\sffamily]
    http://<host where engine is running>:<port>/stop
\end{lstlisting}

\noindent Per default, the engine uses 4812, so you could use the
following URL to stop that engine on the local computer:

\begin{lstlisting}[frame=none,keywordstyle=\sffamily]
    http://localhost:4812/stop
\end{lstlisting}

% ---------------------------------------------------------------------------
\section{\INDEX{Engine Web Server}} \label{sec:enginehttp} % ----------------
\begin{figure}[htb]
\begin{center}
\InsertImage{width=0.8\textwidth}{engine_main}
\end{center}
\caption{\label{fig:engine:main}Main Page of Archive Engine's HTTPD}
\end{figure}

The ArchiveEngine has a built-in web server (HTTP Daemon) for status
and configuration information.  You can use any web browser to access
this web server.  You can do that on the computer where the
ArchiveEngine is running as well as from other computers, be it a PC
or Macintosh or other system as long as that computer can reach the
machine that is running the ArchiveEngine via the network.  You do
\emph{not} need a web server like the Apache web server for Unix or
the Internet Information Server for Win32 to use this. The
ArchiveEngine \emph{itself} acts as a web server.

You \emph{cannot view archived data} with this mechanism.  See the
documentation on data retrieval (chapter \ref{chap:retrieval}) for
that, because the archive engine's HTTPD is meant for access to the
status and configuration of the running engine, not for accessing the
data samples.

To access the ArchiveEngine's web server, you need to know the
Internet name of the machine that is running the ArchiveEngine as well
as the TCP port.  If you are on the same machine, use ``localhost''.
The port is configured when you start the ArchiveEngine, it defaults
to 4812. Then use any web browser and point it to

\begin{lstlisting}[frame=none,keywordstyle=\sffamily]
    http://<host where engine is running>:<port>
\end{lstlisting}

\noindent Example for an ArchiveEngine running on the local machine with the
default port number:

\begin{lstlisting}[frame=none,keywordstyle=\sffamily]
    http://localhost:4812
\end{lstlisting}

\noindent The start page of the ArchiveEngine web server should look
similar to the one shown in Fig.~\ref{fig:engine:main}. By following
the links, one can investigate the status of the groups and channels
that the ArchiveEngine is currently handling. The ``Config'' page
allows limited online-reconfiguration. Whenever a new group or channel
is added, the engine attempt to write a new config file called
\INDEX{onlineconfig.xml} in the directory where is was started.
It is left to the user to decide what to do with this file: Should it
replace the original configuration file, so that online changes are
preserved? Or should it be ignored, because online changes are only
meant to be temporary and with the next run of the engine, the
original configuration file will be used?

Note also that the ArchiveEngine does not allow online removal of
channels and groups. The scan mechanism of a channel can only be
changed towards a higher scan rate or lower period, similar to the
handling of multiply defined channels in a configuration file. Refer
to the section discussing the ``period'' tag on page \pageref{sec:period}.

\section{Threads} % ----------------------------------------------------------
The ArchiveEngine uses several threads:
\begin{itemize}
\item A main thread that reads the initial configuration and then
  enters a main loop for the periodic scan lists and writes to the
  disk.
\item The ChannelAccess client library is used in its multi-threaded
  version. The internals of this are beyond the control of the
  ArchiveEngine, the total number of CA client threads is unknown.
\item The ArchiveEngine's HTTP (web) server runs in a separate thread,
  with each HTTP client connection again being handled by its own
  thread. The total number of threads therefore depends on the number
  of current web clients.
\end{itemize}
As a result, the total number of threads changes at runtime. Though
these internals should not be of interest to end users, this can be
confusing especially on older releases of Linux where each thread
shows up as a process in the process list.
On Linux version 2.2.17-8 for example we get process table entries as
shown in Tab.~\ref{lst:aeprocs} for a single ArchiveEngine, connected
to four channels served by excas, no current web client. The only hint
we get that this is in fact one and the same ArchiveEngine lies in the
consecutive process IDs.

\begin{lstlisting}[float=htb,
caption={Output of Linux 'ps' process list command, see text.},
label=lst:aeprocs]
  PID TTY          TIME CMD
29721 pts/5    00:00:00 ArchiveEngine
29722 pts/5    00:00:00 ArchiveEngine
29723 pts/5    00:00:00 ArchiveEngine
29724 pts/5    00:00:00 ArchiveEngine
29725 pts/5    00:00:00 ArchiveEngine
29726 pts/5    00:00:00 ArchiveEngine
29727 pts/5    00:00:00 ArchiveEngine
29728 pts/5    00:00:00 ArchiveEngine   
\end{lstlisting}

The first conclusion is that one should not be surprised to see
multiple ArchiveEngine entries in the process table.
The other issue arises when one tries to 'kill' a running
ArchiveEngine. Though the preferred method is via the engine's web
interface, one can try to send a signal to the first process, the one
with the lowest PID.

