\section{Index Tool} \label{sec:indextool}
The ArchiveIndexTool is used to create Master Indices by combining
multiple indices into a new one.  When invoked without valid
arguments, it will display a command description similar to this:

\begin{lstlisting}[frame=none,keywordstyle=\sffamily]
USAGE: ArchiveIndexTool [Options] <archive list file> \
                                         <output index> 
Options:
  -help             Show Help
  -M <3-100>        RTree M value
  -verbose <level>  Show more info
\end{lstlisting}

\noindent The archive list file lists all the sub archives,
that is the paths to each sub-archive's index file. It needs to be an
XML file conforming to the DTD in listing \ref{lst:indexconfigdtd}
(see section \ref{sec:dtdfiles} on DTD file installation).
Listing \ref{lst:indexconfigex} provides an example.

\lstinputlisting[float=htb,keywordstyle=\sffamily,caption={XML DTD for
    the Archive Index Tool Configuration},label=lst:indexconfigdtd]{../IndexTool/indexconfig.dtd}

\lstinputlisting[float=htb,keywordstyle=\sffamily,caption={Example
    Archive Index Tool Configuration},label=lst:indexconfigex]{../IndexTool/indexconfig.xml}

We refer to chapter \ref{ch:examplesetup} for an example of how to use
the ArchiveIndexTool in collaboration with the other Channel Archiver
tools. As an aid to creating configuration files for the
ArchiveIndexTool, you can use the perl script ``make\_indexconfig.pl''
that converts a list of index files into the appropriately formatted XML:
\begin{lstlisting}[frame=none,keywordstyle=\sffamily]

USAGE: make_indexconfig [-d DTD] index { index }
 
This tool generates a configuration for the ArchiveIndexTool
based on a DTD and a list of index files provided via
the command line.
\end{lstlisting}
