\chapter{Common Errors and Questions}

The following explains error messages and commonly
asked questions.

\noindent
\textbf{Why is there no data in my archive?}\\
The ArchiveEngine should report warning messages whenever the
connection to a channel goes down or when there is a problem with the
data. So after a channel was at least once available, there should be
more or less meaningful messages. After the initial startup, however,
there won't be any information until a channel is at least once
connected. So if a channel never connected, the debugging needs to
fall back to a basic CA error search:\\
Is the data source available?
Can you read the channel with other CA client tools
(probe, EDM, cau, caget, ...)?
Can you do that from the computer where you are running
the ArchiveEngine? Does it work with the environment settings and user
ID under which you are trying to run the ArchiveEngine?

\noindent
\textbf{Why do I get \#N/A, why are there missing values in my spreadsheet?}\\
There are two main reasons for not having any data: There might not
have been any data available because the respective channel was
disconnected or the archive engine was off. When you look at the
status of the channel, those cases might reveil themselves by status
values like ``Disconnected'' or ``Archive Off''.
The ArchiveEngine might also have crashed, not getting a chance to
write ``Archive Off''. That would be a likely case if no channel has
data for your time range of interest.
The most common reason for missing values, however, simply results
from the fact that we archive the original time stamps and you are
trying to look at more than one channel at a time. See the section on
time stamp correlation on page~\pageref{sec:timestampcorr}.  

\noindent
\textbf{Back in time?}\\
The archiver relies on the world going forward in time. When
retrieving samples from an archive, we expect the time stamps to be
monotonic and non-decreasing. Time stamps going back in time break the
lookup mechanism. Data files with non-monotonic time stamps are
useless. Unfortunately, the clocks of IOCs or other computers running
CA servers can be mis-configured. The ArchiveEngine attempts to catch
some of these problems, but all it can do is drop the affected
samples, there is no recipe for correcting the time stamps.

Bottom line: When you receive time-stamp related warnings, it is too
late. You need to have the clocks of all CA servers properly configured
(also see page~\pageref{back:in:time}).

\noindent
\textbf{Found an existing lock file 'archive\_active.lck'}\\
When the ArchiveEngine is started, it creates a \INDEX{lock file} in
the current directory. The lock file is an ordinary text file
that contains the start time when the engine was launched. When
the engine stops, it removes the file.

The idea here is to prevent more than one archive engine to run
in the same directory, writing to the same index and data files
and thus creating garbage data: Whenever the archive engine sees
a lock file, it refuses to run with the above error message.

Under normal circumstances, one should not find such lock files
left behind after the engine shuts down cleanly. The presence of
a lock file indicates two possible problems:
\begin{enumerate}
\item[a)] There is in fact already an archive engine running in
this directory, so you cannot start another one.
\item[b)] The previous engine crashed, it was stopped without
opportunity to close the data files and remove the lock file.
It \emph{might} be OK to simply remove the lock file and try
again, but since the crash could have damaged the data files, it
is advisable to back them up and run a test before removing the
lock file and starting another engine.
\end{enumerate}

\noindent
\textbf{'ChannelName': Cannot add event because data type is unknown}\\
The ArchiveEngine tried to write an \INDEX{event} to the data
file. Examples include a ``Disconnected'' or ``Archiver Off''
event. Even though this event does not have a value, it only
indicates a status change or warning, it nevertheless is written
into the same data buffer where ordinary values are written.
A problem arises when we never got a connection to the CA
server, therefore we do not know the value type of the channel
and thus we cannot allocate a data buffer in which to write this
special event-type of value.

