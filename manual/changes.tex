\chapter{Changes}

This chapter describes the version numbers and changes since
the beginning of the R3.14 port.

% Compare: make.cfg, ArchiveDaemon.pl $version
\begin{itemize}

\item ?/?/2006 --- Version 2.9:\\
During restarts, ArchiveDaemon.pl waits for an engine that was stopped
to remove its lock file before starting a new one.

\noindent Major refactoring of the ArchiveEngine code.

Visible changes:
\begin{itemize}
\item Engine HTTPD displays ``Idle Time'' on its main page.
\item The warning ``Cannot add event because data type is unknown'' is gone.
\item The ``disabling'' mechanism now works better right at the start,
      while all the channels get connected in an unpredictable order.
      Before, the engine would like to log all disabled channels as DISABLED,
      but couldn't for those which were not yet connected,
      resulting in the ``Cannot add event...'' warning and no indication
      in the data as to why there is no data.
\item ArchiveEngine uses about 2\%-4\% more CPU and memory.
\end{itemize}

Why? The class layout looked OK from a distance;
there was an Engine, Group, ArchiveChannel, SampleMechanism,
the latter derived into SampleMechanismMonitor etc.
But while the SampleMechanism was supposed to handle
the scanning, the ArchiveChannel held the Circular buffer
into which the samples are placed,
and the Engine handled the ChannelAccess 'flush' required
for a 'get'. So there were several messy interactions,
which made it impossible to test and debug for example a SampleMechanism
as a Unit outside of the complete Engine.

In the rework, more classes were added 'below' the SampleMechanism:
ProcessVariable, ProcessVariableFilter, ProcessVariableContext,
and 'Listener' interfaces for these and the existing classes.
This way, the ProcessVariable can be tested without the Engine,
the sample mechanisms can be tested with test data,
and the rather confusing code inside the sample mechanisms
has mostly been replaced with a 'pipe' of ProcessVariableFilter
classes, which again can be tested individually.

Also switched to using Eclipse as the IDE.
Doesn't nearly work as nicely for C++ as it does for Java,
but still nice.
Basic setup:
\begin{itemize}
\item Unpack ChannelArchiver sources as usual.
\item Open Eclipse, use the directory ``extensions/src'' as a
      workspace directory.
\item Create a new Standard C++ Makefile Project, using
      ``ChannelArchiver'' as the name. It will parse the existing
      sources, but with a few indexer errors.
\item In ``Project/Properties/C/C++ Include Path and Symbols'',
      select ChannelArchiver, ``Add Include .. from Workspace'',
      and add Tools, Storage, LibIO, XML-RPC, Engine.
\end{itemize}

\NOTE When running under Fedora Core 2 (Linux kernel 2.6.5,
gcc 3.3.3), the ArchiveEngine leaked memory whenever its HTTPD
was accessed. The memory leak was visible in the ``top'' command,
and in fact the engine would crash after some time with out-of-memory messages
from pthread\_create.

Interestingly, ``valgrind'' did not indicate a \INDEX{memory leak} on that
same architecture. Furthermore, I could also not detect a memory leak on RedHat WS 4
nor Mac OS X, so I assume this was a freak behavior for only Fedora 2,
some combination of its compiler and runtime library. 

\item 03/27/2006 --- Version 2.8.1:\\
Fix: The nanosecond check could cause a crash when receiving
CA callback for 'disconnected' values.

\item 03/24/2006 --- Version 2.8.0:\\
Possibly the most visible change is in the ExampleSetup.
Before, the ``ArchiveDaemon'' would run engines and perform indexing.
The indexing has been moved into a different collection of scripts,
meant to run on a ``serving'' computer, while the ``sampling'' machine
runs only daemons and engines.
Check the ``Example Setup'' in chapter \ref{ch:examplesetup} for details.
It is probably not fully done, but has been running at the SNS for
about two weeks, recently collecing 4 to 5GB of data per day,
and serving a total of about 900GB of data for 2005 and 2006.

The code has been changed back to using C++ exceptions, as it did in
the versions for EPICS base R3.13.

In this conversion process, many unit-tests have been added
and checked under valgrind.

When compiled with R3.14.8, the more rigorous checking of time stamps
in EPICS base would result in 'assert'-aborts when the
nanosecond-portion of time stamps was not normalized.
This happened quite often at the SNS because of ongoing timing system driver
development and last not least a bug in the Java data viewer.
Guards have been added to the archiver code to hopefully avoid all
such aborts.
The Engine will ignore samples with problematic nanoseconds,
while the retrieval tools will replace those nanoseconds with 0.

The retrieval code used by ArchiveExport and the ArchiveDataServer can
now follow several soft links. So if /a/link is a soft link to /b/link
which in turn points to /c/index which refers to data in
``20060110'', the tools will now use the data from 
``/c/20060110'' and not ``/a/20060110''.

Some of the messages logged by the engine are now throttled
to reduce the log file size.

\item 12/13/2005 --- Version 2.7.0:\\
Error reporting: When a command-line tool failed,
one could see a rather specific error message
like "Cannot open DataFile 20040323".
It also ended up in the log file of the network
data server, but was not reported well via
the XML-RPC protocol.

Now an XML-RPC fault with the more specific error
message is returned when an index or data file
access fails.
A request for a channel that is not found at all is now
also considered an error by the network data server.
When there is simply no data for the requested time,
the meta information is returned without any
actual samples, as before.

\item 10/31/2005 --- Version 2.6.0:\\
The toolset now compiles and runs on a 64~bit Linux system,
specifically RedHat Enterprise Linux~WS release~4
with gcc~3.4.4.

The goal here was to keep binary compatibility,
so that index and data files can be moved from
32~bit computers to 64~bit machines and back.
This means that the 32~bit limititations on
file sizes still apply, even when running
on a 64~bit computer.

Several new self-tests were added to check
the following on the 64~bit system as well
as 32~bit Linux and Mac OS~X 10.3:
\begin{itemize}
\item All compiles with the 64~bit compiler, no warnings on the test system.
\item 'Tools' library self-test passes.
\item 'Storage' library self-test passes.
\item ArchiveExport can dump demo data,
      matching results on all architectures.
\item ArchiveDataTool can copy demo data,
      matching results on all architectures.
\item DataServer runs from within shell script
      that simulates a CGI environment.
\item ArchiveEngine runs, HTTPD responds,
      collected data passes simple tests.
\end{itemize}

\item 09/21/2005 --- Version 2.5.0:\\
New engine URL
\begin{verbatim}
.... castatus?/tmp/x
\end{verbatim}
will cause the engine to dump the ca\_client\_status() output
into the given file the next time it writes to the disk.

\item 08/29/2005 --- Version 2.4.0:\\
LinearReader rounds start time,
because otherwise a request for multiple channels,
all using linear interpolation, could loose
lockstep, and when the result is put in a spreadsheet
form, there are many more lines than expected
from the per-channel view of the data,
which confused the data viewer.

XML-RPC, being XML, sends all numbers as strings.
Unfortunately, XML-RPC also insists in the simple 'dot'
notation and prohibits exponential notation.
A number like 1e-300 would turn into
"0.0000000..." with 300 zeroes,
which is too long for the internal print buffer of
the xml-rpc C library.
Since such huge and tiny numbers can't be transferred,
they are replaced by 0 with stat/sevr UDF/INVALID.

The cut-off point is somewhat arbitrary.
The XML-RPC library uses an internal print buffer
of about 120 characters.
Since PVs are usually scaled to be human-readable,
with only vacuum readings using exp. notation for
data like "1e-8 Torr", exponents of +-50 seemed
reasonable.

\item 07/18/2005 --- Version 2.3.0:\\
Added ArchiveDataServerStandalone.

\item 05/31/2005 --- Version 2.2.1:\\
At the SNS with EPICS R3.14.7, there is an instance of an engine crashing in
ipAddrToAsciiEnginePrivate.cpp in the transactionComplete callback while
pCurrent was 0, which shouldn't happen:
\begin{verbatim}
if (!this->pCurrent)
    continue;
{
    epicsGuardRelease < epicsMutex > unguard ( guard );
     this->pCurrent->pCB->transactionComplete ( this->nameTmp );
}
\end{verbatim}

In the process of debugging it, an uninitialized use of the HTTPD run duration
variable and a scanlist memory leak on exit have been found by 'valgrind'.
None of this should matter in the real world, but I prefer as little valgrind
complaints as possible. Remaining memory leaks on exit are in the EPICS base
code.

\item 03/31/2005 --- Version 2.2.0:\\
Added a "Write Duration" status to the main
page of the Engine HTTPD.
It is a running average of the time spent
writing to the disk.
ExampleSetup includes a new script,
"engine\_write\_durations.pl",
which queries all engines for their write duration.
The intent is to get an idea of how much time
is spent doing disk I/O, and to get a grasp on
which engine is doing worst.

\item 03/21/2005 --- Version 2.1.9:\\
The ArchiveDaemon.pl script now creates a soft-link
"current\_index" in the engine subdirectories,
pointing to the currently used index.
The idea is to have a common name for the active index
in case the multi-index is broken or only updated
slowly.

The next issue was that the relation of an index
to its data files must not change.
If an index is in the same directory as its data files,
OK, but this soft link was actually in a different location,
so in order to still find the data files, the retrieval
now follows sym-links to index files on level deep
(not following arbitrary chains of soft links).

\item 02/01/2005 --- Version 2.1.8:\\
At the SNS, one ArchiveEngine running a configuration with
'disabling' channels tended to hang up just after a channel
'disabled' a group.
While I was never able to capture this in the debugger or
otherwise reproduce it, I did find a violation of the lock
order in the code related to 'disabling' a group, so this
might have been the reason.

\item 10/26/2004 --- Version 2.1.7:\\
In monitored mode, engine will always try to add two initial
samples: One with the original time stamp, one with the host
time stamp. This solves the restart problem:

Assume a setpoint PV is 3 days old. When we start an engine today,
it'll write that 3-day old value. When we then stop and restart tomorrow
in a new directory, a "Disconnected" value will be added before shutdown
and then the same happens in the new directory.
When now looking at both days, you would get the 3-day old value,
a disconnect/off value and nothing else.

With this update, you will see one value with the host's time stamp
each day in addition to the original time stamp, so even when viewed
via a master-index, one sample per day should be visible.

ArchiveDaemon.pl no longer complains about an undefined "opt\_i",
thanks to Paul Sichta for pointing this out.

\item 09/07/2004 --- Version 2.1.6:\\
Added Enable/Disable code to ArchiveDaemon.
``ListIndex'' bug fixes (sub-index was
closed even though we're still accessing channels in the index).

\item 08/26/2004 --- Version 2.1.5:\\
Fixed bug in the engine code: PVs which never change
were never written when sampling with period < get\_threshold.

Working on a ``ListIndex'', allowing the retrieval
tools to use an indexconfig.dtd-type list of
sub-archives, querying them one by one.
The result is slower than using a ``real'' index,
but much easier to setup and maintain.

\item 07/26/2004 --- Version 2.1.4:\\
Bug in DataReader that affected all the retrieval code:
The ``find'' uses the start time in an at-or-before sense,
which is intentional for direct calls to find().
This happened, too, when it was used internally for the purpose
of switching to a new data block ref'ed by a (master) index.
In that case, however, the start time given by the index must
be observed in an at-or-after sense, otherwise we can go
back in time.

Added the weekly option to the ArchiveDaemon.
Not detailed in manual until we get some milage on it.

\item 07/23/2004 --- Version 2.1.3:\\
Index update had a flaw:
When an engine stops, that last value received via CA
might have a time stamp of 10:00 and then we stop
at 11:00, so the last stored value is the ``off'' value
at 11:00.
When now a new engine starts, the first data is still
10:00, and that data is hidden under the last data block
of the previous engine (...11:00).
When then the new engine added more data, eventually
beyond 11:00, that new data stayed hidden unless
one rebuilt the master index from scratch.
Hopefully fixed this.

\item 06/30/2004 --- Version 2.1.2:\\
Added tags and channel names to the Data files,
so that in the future one could try to write
a rescue tool.
Patch for XML-RPC C/C++ lib.\ allows small numbers.
DataTools' ``index2dir'' option now actually works.
Matlab/Octave glue code can handle an array channel
(when requesting a single channel, raw data).
ArchiveDaemon generates indexupdate.xml for each re-index run;
``-u'' option.

When retrieval uses a "master" index and reaches
its end, it will try to continue by following
links of the last data block in the sub-archive.
This will allow us to get closer to "now" between
updates of the master index.
The first attempt to implement this failed because
of errors in the handling of the path names
(master index has path to data files, but when
we follow the links inside the sub-archive's data
file, those are relative to where the sub-archive resides).
Still not fully tested.

\item 04/01/2004 --- Version 2.1.1:\\
Many little updates have been checked into CVS
while the tools reported "2.1.1".
In the end, the engine supported sampling, monitoring
and sampling-based-on-monitors, and ran without known problems
under valgrind.
The XML-RPC Data Server seemed to work fine, supporting
raw data, plot-binning, spreadsheets, averaged and linear interpolation.
Java Archive client is useable.
Switched index file to CAI2, where the name hash includes a filename
for the RTree. For now it's left empty, but the file format now
allows for a further extension where certain RTrees are in separate
files as soon as the index file gets too big.

\item 01/27/2004 --- Version 2.1.0:\\
Uses new RTree, initial XML-RPC Data Server, XML configuration files.

\item 09/05/2003 --- Version 2.0.1:\\
Some bug fixes:
The  "scanned"  operation didn't work, and when all was monitored,
the  empty  scan lists lead to a high CPU load. Still not perfect,
the  ChannelInfo  code  should  be  split  into  really monitored,
scanned using CA monitor and scanned using CA get.

\item 04/04/2003 --- Version 2.0:\\
Starting to work on R3.14 port.
\end{itemize}
