\chapter{Overview}
The Channel Archiver is an archiving toolset for the Experimental Physics
and Industrial Control System
(\INDEX{EPICS}, see
 \href{http://www.aps.anl.gov/epics/}{http://www.aps.anl.gov/epics/}).
It can archive any value that is available via \INDEX{ChannelAccess}
(\INDEX{CA}), the EPICS network protocol. We use the term ``archiver''
whenever we refer to the collection of programs which allow us to take samples,
place them into some storage and retrieve them again. The archiver
toolset roughly splits into the following pieces:

\begin{description}
\item[\sffamily Sampling:]
The ArchiveEngine collects data from a given list of ChannelAccess
Channels.  The details of when a sample is taken etc.\ can be
configured: One may store every change, store changes that exceed a
deadband (that is configured on the CA server) or use periodic
scanning.
The configuration and operation of the ArchiveEngines will obviously
require some planning, as only data that was sampled and stored will
be available for future retrieval and analysis. Some sensible
compromise will have to be made between the urge to store all
minuscule changes of all the available channels of a site on one hand
and data storage constraints on the other.

\item[\sffamily Storage:]
The data is stored in binary index and data files. Most end users need not
be concerned about the internals of those files, not even where they
are located, because additional indices allow several sub-archives to
appear like one, bigger, combined archive.
Somebody at each site, though, will need to perform maintenance
tasks: Decide where the data sets are located, how they are
backed up and how users can access them. 

\item[\sffamily Retrieval:]
The archiver toolset provides generic retrieval tools for browsing the
available channels and values, including simple multi-channel
comparisons.
An API allows users to write more sophisticated data analysis tools.
\end{description}

