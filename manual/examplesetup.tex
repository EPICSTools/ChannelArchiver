% ======================================================================
\chapter{Example Setup} \label{ch:examplesetup}
% ======================================================================
The following describes how the archiver toolset is used at the
\INDEX{Spallation Neutron Source (SNS)}, and the scripts in
the ChannelArchiver/ExampleSetup directory assist with the
setup as described in here.
We distinguish between two types of computers:
\begin{itemize}
\item Sampling Computer:\\
      A computer that runs ArchiveEngine instances.
\item Serving Computer:\\
      A computer that uses the ArchiveIndexTool to create
      additional binary indices and runs the ArchiveDataServer.
\end{itemize}

\noindent There might be more than one 'sampling' computer as well as more
than one 'serving' computer.
A single machine might perform both functions, but in general
they can be different computers on the network, and hence
some tools are required to make the data collected on the
"sampling" computer available on the "server".
One could use NFS, but we have decided to use secure copy
(scp) in order to decouple the computers as best as possible.

We want to be able to move an engine from one computer to another,
and still keep an overview.
Therefore a file ``/arch/archiveconfig.xml'' describes the complete
archive setup: Which engines run where, and how the data gets served.
On some computers, for example ics-srv-archive1, further subdirectories of ``/arch''
are used to run engines.
On another computer, for example ics-srv-web2, subdirectories
contain data copied from archive1 so that the data server
can serve it.

% ======================================================================
\section{archiveconfig.xml}
% ======================================================================
Each computer needs to have the same copy of /arch/archiveconfig.xml.
You might generate and distribute that file manually
or use a relational database.
People who have used previous releases of the archive toolset might
remember the archiveconfig.csv file. There is a tool
convert\_archiveconfig\_to\_xml.pl to convert that file
into an archiveconfig.xml skeleton.

\lstinputlisting[float=htb,keywordstyle=\sffamily,caption={archiveconfig.xml},label=lst:archiveconfig]{../ExampleSetup/archiveconfig.xml}

\noindent The archiveconfig.xml file describes the complete archive layout,
using the following elements:
\begin{itemize}
\item root: Names the root directory, typically '/arch'.
      This has to be the same directory name on all computers,
      since they all use the same archiveconfig.xml.
\item serverconfig: Location of the server configuration to be
      created. See section \ref{sec:updateServer}.
\item mailbox: Used to communicate from the ArchiveDaemons to
      the data server.
\item daemon: Configures an archive daemon and its engines,
      described in section \ref{sec:daemon},
      as well as how that data should be indexed and served,
      see sections \ref{sec:updateServer} and \ref{sec:updateIndices}.
\end{itemize}

% ======================================================================
\section{"Sampling" Computer}
% ======================================================================
\begin{figure}[htb]
\begin{center}
\InsertImage{width=0.9\textwidth}{archiveconfig_sample}
\end{center}
\caption{\label{fig:acSample}Tools used on a sampling computer, refer to text.}
\end{figure}

\noindent Instead of running just one archive engine, it is often preferred to
run several, and even restart them weekly. This way, people working on different
subsystems can each use configure engines for only their systems without interfering
with each other.
The periodic restarts, creating different sub-archives for example each week,
limit the possibility for data loss in case an engine crashes or creates
corrupt archives.

For each subsystem (examples: Vacuum, Cooling, ...), a "Daemon" program
manages one or more engines.
Initially and after every change to the configuration,
the update\_archive\_tree script is used to create the necessary infrastructure.

% ======================================================================
\subsection{update\_archive\_tree.pl}
% ======================================================================
This script will read archiveconfig.xml and create all the subdirectories,
daemon config files, and skeleton engine configurations.
Run it with "-h" to see available options.

% ======================================================================
\subsection{ArchiveDaemon} \label{sec:daemon}
% ======================================================================
ArchiveDaemon.dtd
ArchiveDaemon.pl
ArchiveDaemon.xml

start\_daemons.pl
stop\_daemons.pl

\begin{figure}[htb]
\begin{center}
\InsertImage{width=0.8\textwidth}{daemon}
\end{center}
\caption{\label{fig:daemon}Archive Daemon, refer to text.}
\end{figure}

\noindent The ArchiveDaemon is a script that automatically starts,
monitors and restarts ArchiveEngines on the local host. It includes a
built-in web server, so by listing all the ArchiveEngines that are
meant to run on a host in the ArchiveDaemon's configuration file, one
can check the status of all these engines on a single web page as
shown in Fig.~\ref{fig:daemon}.

Chapter \ref{ch:examplesetup}, ``Example Setup'', on page
\pageref{ch:examplesetup}, gives more details on the suggested use of
the ArchiveDaemon. The daemon will attempt to start any ArchiveEngine
that it does not find running. In addition, the daemon can
periodically stop and restart ArchiveEngines in order to create
e.g.\ daily sub-archives.  Furthermore, it adds information about
each sub-archive of newly created ArchiveEngines to a mailbox directory
so that the index mechanism can create the necessary indices
and update the data server configuration.

Before using the ArchiveDaemon, one should be familiar
with the configuration of the ArchiveEngine (sec.\ \ref{sec:engine}),
and how to start and stop it. Furthermore, one needs to be familiar
with the ArchiveIndexTool (sec.\ \ref{sec:indextool}).

\subsection{Configuration}
The ArchiveDaemon expects to find a configuration file called
``ArchiveDaemon.xml'' in the directory where it is started.  That
configuration file, an example of which can be found in listing
\ref{lst:daemonconfigex}, needs to follow the DTD from listing
\ref{lst:daemonconfigdtd}.

In many cases you might not configure the daemon yourself,
but instead use the mechanism described in chapter \ref{ch:examplesetup}.

\lstinputlisting[float=htb,keywordstyle=\sffamily,caption={Example Archive Daemon Configuration},label=lst:daemonconfigex]{../ExampleSetup/ArchiveDaemon.xml}

\lstinputlisting[float=htb,keywordstyle=\sffamily,caption={XML DTD for
    the Archive Daemon Configuration},label=lst:daemonconfigdtd]{../ExampleSetup/ArchiveDaemon.dtd}

\noindent The configuration lists all the
ArchiveEngines that the daemon should manage on the local
computer. One ``engine'' element per ArchiveEngine specifies the
configuration of each engines. Specifically, the following tags are
allowed:

\subsubsection{desc}
This mandatory element is used for the ``-description'' option of the
Archive Engine, see section \ref{sec:enginedesc}.

\subsubsection{port}
This mandatory element determines the port number of the engine's HTTP
server, see section \ref{sec:engineport}.

\NOTE The ArchiveDaemon itself requires a TCP port number for its HTTP
server. The port numbers used by the ArchiveDaemon and all the Archive
Engines need to be different. You cannot use the same port number more
than once per computer.

\subsubsection{config}
This mandatory element must contain the path to the configuration
file of the respective ArchiveEngine, see section \ref{sec:engineconfig}.
This can be either the full path to the engine configuration file or a path
beneath the current working directory in which the ArchiveDaemon is running.

\subsubsection{daily}
This optional element configures the ArchiveDaemon to restart the
ArchiveEngine periodically. The element must contain a time in the
format ``HH:MM'' with 24-hour hours HH and minutes MM. One example
would be ``02:00'' for a restart at 2~am each morning.

\subsubsection{weekly}
Weekly is similar to daily, but using an element that contains the day
of the week (Mo, Tu,  We, Th, Fr, Sa, Su) in addition to the time
on that day in 24-hour format, e.g.\ ``We 08:00''. In this example,
the daemon will attempt a restart every Wednesday, 8'o clock in the morning.

\subsubsection{timed}
In this case, the element needs to contain a start/duration time pair
in the format ``HH:MM/HH:MM''. The first, pre-slash 24-hour time stamp
indicates the start time, and the second 24-hour time, trailing the
slash, specifies the runtime. The engine will be launched at the
requested start time and run for the duration of the runtime. As an
example, ``08:00/01:00'' requests that the daemon starts the engine at
08:00 and stops it after one hour, probably around 09:00.

\subsubsection{hourly}
This optional element configures the ArchiveDaemon to restart the
ArchiveEngine periodically. The element must contain a number
specifying hours: A value of 2.0 will cause a restart every 2
hours. The hourly restart is quite inefficient and primarily meant for testing.

\subsection{Starting and Running}
The ArchiveDaemon is a perl script that is typically started like this:

\begin{lstlisting}[keywordstyle=\sffamily]
$ cd whereever_you_placed_ArchiveDaemon.xml
$ perl ArchiveDaemon.pl -f ArchiveDaemon.xml
Read ArchiveDaemon.xml, will disassociate from terminal
and from now on only respond via
          http://localhost:4610
You can also monitor the log file:
          ArchiveDaemon.log
\end{lstlisting}

\noindent One can use any web browser to connect to the daemon's HTTP server
under the URL shown in the above status message. Fig.~\ref{fig:daemon}
shows one example. The ArchiveDaemon offers a command line option for
selecting a specific TCP port.
Whenever running more than one ArchiveDaemon per computer, they
need to be started with different TCP port numbers. Furthermore, each
ArchiveEngine needs a different TCP port number.
\begin{lstlisting}[keywordstyle=\sffamily]
USAGE: ArchiveDaemon [options] 
Options:
  -p <port>    : TCP port number for HTTPD
  -f <file>    : config. file
  -i <URL>     : path or URL to indexconfig.dtd
  -u <minutes> : Update period for master index
  -d           : debug mode (stay in foreground etc.)
\end{lstlisting}

\noindent The first few lines of the ArchiveDaemon.pl script contain
numerous configuration variables. They allow fine tuning of e.g.\ how
often the daemon queries the Archive Engines and other customization
options. In there you can also change many of the file names from
their defaults up to the point where none of the following
applies. With the original settings, the ArchiveDaemon will create or
use the following files in the directory in which it was started:
\begin{itemize}
\item ArchiveDaemon.log\\
  The log file of the ArchiveDaemon.
\item indexconfig.xml\\
  This is a configuration file for the ArchiveIndexTool.  If the file
  already exists, the ArchiveDaemon will add every new sub-archive
  that it creates to the file. If the file does not exist, one will be
  created the next time a new sub-archive is started.

  Note that the ArchiveDaemon will \emph{not} search for sub archives
  or index files and automatically add them to indexconfig.xml. It
  will only add newly created sub-archives. If you already have
  sub-archives that need to be included in the master index, your
  initial indexconfig.xml needs to list them.
  Also note that the ArchiveDaemon reads this file on startup. In
  order to \emph{remove} indices from indexconfig.xml, it is therefore
  required to stop the running daemon, since it will otherwise
  overwrite indexconfig.xml with its in-memory version.
\item indexupdate.xml\\
  This file is similar to indexconfig.xml, except that the index files
  for sub-archives that are older than the master index file are
  commented out, assuming that the master index already contains all
  the information from those older sub-archives.
  Running ArchiveIndexTool with this indexupdate file is naturally faster
  than using indexconfig.xml. The ArchiveDaemon uses the full index
  configuration from indexconfig.xml once after startup, and then
  switches to indexupdate.xml.
\item ArchiveIndexTool.log\\
  The log file of the last run of the ArchiveIndexTool.
\item master\_index\\
  The ArchiveIndexTool is run with indexconfig.xml to update
  this master index file.
\end{itemize}

\noindent The ArchiveDaemon configuration file must list the full path
names to the configuration files for the ArchiveEngines or use a path
that is below the current working directory of the ArchiveDaemon.
Within each of those directories, an ArchiveEngine is run and the
following files will be created:
\begin{itemize}
\item ArchiveEngine.log\\
  A log file for the ArchiveEngine running in that directory
\item archive\_active.lck\\
  Lock file of the ArchiveEngine
\item YYYY/MM\_DD/index \\
  A subdirectory for index and data files of the sub-archive.  If the
  ArchiveDaemon is configured to perform daily restarts, the format
  uses the year, month and day to build the path name.
\end{itemize}

\subsection{Daemon's Web Pages}
The main web page of the ArchiveDaemon's HTTPD looks similar to
Fig.~\ref{fig:daemon}. You can use any web browser to look at the
daemon's web pages. The URL follows the format
\begin{lstlisting}[keywordstyle=\sffamily]
    http://host:port
\end{lstlisting}
\noindent where ``host'' is the name of the computer where the
ArchiveDaemon is running. More often than not you will use
``localhost''. ``Port'' is the TCP port that was specified as a
command-line option to the ArchiveDaemon program, otherwise it
defaults to 4610. So the default URL will be http://localhost:4610.

The main page lists all the archive engines that this daemon controls
with their status. The first column also contains links to the
individual archive engines. The status shows any of the following:

\begin{itemize}
\item ``N/M channels connected''\\
      This means the ArchiveEngine is running and responding,
      telling us that N out of a total of M channels have connected.
      If not all channels could connect, you might want to follow
      the link to the individual engine to determine what channels are
      missing and why: Is an IOC down on purpose? Is an IOC
      disconnected because of network problems? Does a channel simply
      not exist, i.e.\ the engine's configuration is wrong?
\item ``Not Runnnig''\\
      This means that the respective ArchiveEngine did not respond
      when we queried it, and there is no ``archive\_active.lck'' lock
      file. This combination usually means that the engine is really
      not running (except for the Note below).

      The first step in debugging would be to check the engine's
      directory for a log file. Does it indicate why the engine could
      not start? Then check the daemon's log file. It should list the
      exact command used to start the engine. You can try that
      manually to check why it didn't work.
\item ``Unknown. Found lock file''\\
      This means that the respective ArchiveEngine did not respond
      when we queried it, but there is an ``archive\_active.lck'' lock
      file. This could have two reasons. It could mean that the engine
      is running but it was temporarily unable to respond to the
      daemon's request. An example would be that the engine is really
      busy writing and dealing with ChannelAccess, so that its web
      server had to wait and the daemon timed out. All should be fine
      again after some time.

      If, on the other hand, the situation persists, it usually means
      that the engine is hung or has crashed, so that it does not
      respond and the lock file was left behind.
      See Crashes on page \pageref{sec:crash}.
\end{itemize}

\NOTE The daemon queries only every once in a while and leave the
engines alone for most of the time.
Especially after startup, all engines will show up as ``Not Running''
in the daemon's web page while in fact most of them are already
running. Then you will see many disconnected channels while the
engines did in fact already connect to all channels. 
If you are impatient, you can click on the links to the individual
engines to get a more up-to-date snapshot of each engine's status.

\subsection{Disabling Engines}
The web interface of the daemon contains a link for each engine that
disables the engine. This places a file ``\INDEX{DISABLED.txt}'' in
the engine directory and stops the engine.
The daemon will not attempt to start engines as long as the ``DISABLED''
file is found. This is a convenient way for temporarily disabling
engines without removing them from the daemon's configuration.

\subsection{Stopping and More}
To stop the ArchiveDaemon, access the ``/stop'' URL of the daemon's
HTTPD, e.g. ``http://localhost:4610/stop''.  Similar to the
ArchiveEngine's HTTPD, this URL is not accessible by following links
on the HTTPD's web pages. You will have to type the URL. This is to
prevent web robots or a monkey who is sitting in front of the computer
and clicking on every link from accidentally stopping the daemon.
Finally, the daemon will respond to the URL ``/postal'' by stopping every
ArchiveEngine controlled by the daemon, followed by stopping itself.

%----------------------------------------------------
%----------------------------------------------------


\subsection{Status Information}
engine\_write\_durations.pl
make\_archive\_infofile.pl
make\_archive\_web.pl

show\_engines.pl
show\_sizes.pl

\section{Indices}
update\_indices.pl

\section{Data Server}
update\_server.pl


ALL THE REST NEEDS TO BE REVIEWED!

We intend to run one ``ArchiveDaemon'' for the Integrated Control
System (ICS) and one for the channels related to Radio Frequency (RF).
ArchiveDaemon, described in full detail in chapter \ref{sec:daemon},
starts archive engines and monitors their operation. The ICS daemon
should maintain one ArchiveEngine for the timing system (tim) and one
for the machine protection system (mps), while the RF daemon has one
engine for the low-level and one for the high power RF (llrf, hprf).

This separation is somewhat arbitrary. We could have made ``llrf'' and
``hprf'' channel groups under one and the same engine. In fact all the above
could reside within one engine. It is, however, advisable to spread the channels
over different daemons and engines whenever different people deal with
the IOCs that host the channels, so that the engineers can
independently configure their archiving. 
In addition, you want to keep the amount of data collected by each
engine within certain bounds, for example: not more than one CD ROM
per month, one DVD per year, or whatever you plan to do for data
maintenance. Another reason is data safety: You can reduce the damage
caused by crashes of the engine by limiting the number of channels per engine.

\section{Current Archive Status} \label{sec:makearchweb}
The perl script ``\INDEX{make\_archive\_web.pl}'' creates a web page
displaying the current status of all the archive daemons and engines
that are listed in the file ``archiveconfig.csv''
(you could use other names for the main configuration file, but we
will stick with ``archiveconfig.csv'' in this manual).
For the SNS, this script is periodically executed with the result being
redirected into the web server's document tree, so you can usually
reach the recent status from this web page:
\begin{lstlisting}[frame=none,keywordstyle=\sffamily]
    http://ics-srv-archive1/archive
\end{lstlisting}
\noindent  Fig.~\ref{fig:archcfgstat} shows one example of that web
page: A tabular display of what engines are running, how many channels are
connected, and more.

\medskip

\begin{figure}[htb]
\begin{center}
\InsertImage{width=\textwidth}{archcfgstat}
\end{center}
\caption{\label{fig:archcfgstat}Example of the archive status web page
  generated by the make\_archive\_web.pl script.}
\end{figure}

\section{Directory Layout}
The perl script ``\INDEX{make\_archive\_dirs.pl}'' creates or updates a
directory tree based on the file ``archiveconfig.csv''. For the preceding
example, it would create subdirectories ``ICS'' and ``RF'' for the two
daemons, plus the following engine directories:
\begin{lstlisting}[frame=none,keywordstyle=\sffamily]
  ICS/tim
  ICS/mps
  RF/llrf
  RF/hprf
\end{lstlisting}
\noindent Each daemon directory contains a daemon configuration file
as well as start/stop scripts. Each engine directory contains a script
to stop the engine. Note the absence of a script to start the engine:
You should not manually start engines which are under the control of
an archive daemon.  Each engine directory also contains an
``ASCIIConfig'' subdirectory with a script ``convert\_example.sh''
that you might use to create the XML configuration file for the
ArchiveEngine from ASCII configuration files, though the engineer
responsible for the subsystem is free to use any method of his/her
choice as long as the result is a configuration file for the engine
that follows the naming convention
\begin{center}
{\it Daemon-Name} / {\it Engine-Name} / {\it Engine-Name}-group.xml~,\\
\end{center}
resulting in these for our example:
\begin{lstlisting}[frame=none,keywordstyle=\sffamily]
  ICS/tim/tim-group.xml
  ICS/mps/mps-group.xml
  RF/llrf/llrf-group.xml
  RF/hprf/hprf-group.xml
\end{lstlisting}
As long as you end up with engine configuration files of these names,
you can employ any text editor, a copy of the example script, or a
sophisticated toolset utilizing a relational database.
If you want to use the ASCIIConfig directory, check section
\ref{sec:ASCIIConfig} for the format of the ASCII configuration files.

\section{Sub-Archives}
Based on the configuration from the beginning of this chapter,
the daemon will 
\begin{enumerate}
\item Start an ArchiveEngine in ``RF/llrf'' that writes to
      a sub-archive named after the current day,
      e.g.\ ``RF/llrf/2004/02\_11/index''.
      Same for a second engine in ``RF/hprf''.
\item Periodically verify if engines that are supposed to run are
      actually running, attempting to start engines which are found missing.
\item Stop the ArchiveEngines each Wednesday at 10:20 respectively
      10:30 and restart them in new subdirectories
      (using the data of the restart for the path to the index file).  
\item Generate or update ``RF/indexconfig.xml'' whenever a new
      sub-archive is created for the vacuum or cooling data.
\item Periodically run ArchiveIndexTool on ``RF/indexconfig.xml'',
      generating or updating ``RF/master\_index''.
\item Provide a web page that lists the status of the two archive
      engines.
\end{enumerate}

\noindent As a result, we create separate sub-archives for the LLRF
and HPRF, a new one once per week, which provides some insurance
against crashes of an ArchiveEngine. If you are paranoid, you can
choose daily sub-archives; compare the ICS setup from the beginning of
chapter \ref{ch:examplesetup}.
After running for a while, we will have created sub-archives like these:
\begin{lstlisting}[frame=none,keywordstyle=\sffamily]
  RF/llrf/2004/02_11/index
  RF/llrf/2004/02_18/index
  RF/llrf/2004/02_25/index
  ...
  RF/hprf/2004/02_11/index
  RF/hprf/2004/02_18/index
  RF/hprf/2004/02_25/index
  ...
\end{lstlisting}
\noindent In addition to each index file, there will of course also be
associated data files, but for retrieval purposes we identify an
archive solely by its index file: We can invoke e.g.\ ArchiveExport
with the path to any of the index files. This is, however,
inconvenient because we will only see data for one week of that one
subsystem at a time. The periodic invocation of the ArchiveIndexTool
allows us to view all the RF data as a whole via the RF/master\_index.

\subsection{ArchiveDaemon Details}
The ``Restart'' and ``Time'' columns of the ``archiveconfig.csv'' file
are passed on to the ArchiveDaemon, which is explained in more detail
in chapter \ref{sec:daemon}. In many cases it might be sufficient to
know these two options:
\begin{itemize}
\item Restart=``daily'', Time set to the time of day in 24-hour
      format HH:MM, e.g.\ 08:00 for 8'o clock in the morning.
      The daemon will stop and restart the engine in a new sub-archive
      each day at the given time.
\item Restart=``weekly'', Time set to a string that combines the day
      of the week (Mo, Tu,  We, Th, Fr, Sa, Su) with the time of day in 24-hour
      format into ``DD HH:MM'', e.g.\ ``We 08:00'' for Wednesdays, 8'o
      clock in the morning. Similar to the daily setup, but reduced to
      once a week.
\end{itemize}
\noindent It is advisable to stagger the restart times of your engines
such that they don't all restart at the same day and time in order to
reduce the CPU and network load for the ChannelAccess re-connects.

\section{Common Tasks}

\subsection{Check Daemon, Engine, Connected Channels, ...}
See section \ref{sec:makearchweb}. That web page links to all
the ArchiveDaemons, which in turn link to all the ArchiveEngines.

\subsection{Modify Engine's Request Files}
Locate your archive engine on either the main archive status page
(section  \ref{sec:makearchweb}) or in /arch/archiveconfig.csv.
According to the example at the beginning of this chapter, the ``High
Power RF'' engine would be run by the ``RF'' daemon and be called
``hprf'', so we need to modify /arch/RF/hprf/hprf-group.xml.
This is often done via a conversion script in
/arch/RF/hprf/ASCIIConfig. If you used another method to create the
engine configuration, this is a good time to remember what you did.

Then, to actually use that new config file, the engine needs to
restart. We could simply wait for the next scheduled restart, in our
example the next Wednesday, 10:30. Alternatively, we can run the
script /arch/RF/hprf/stop-engine.sh.
Watch the RF daemon via the link on the main archive status page.
Within a few minutes, it ought to detect that the engine had stopped
and then restart it.

\subsection{Add Engine or Daemon}
Add a line to archiveconfig.csv to define the new engine under an
existing demon. Or add a line for a new daemon, then add the new
engine under it. Invoke  make\_archive\_dirs.pl. Per default, it will
re-create all daemon and engine directories, so you might want to use
the ``-s'' option to limit its operation to the new or modified
subsystem.
In case the daemon was already running, it won't learn about the new
engine unless you restart it. So run the ``stop-deamon.sh'' and then
invoke the ``run-daemon.sh'' script in the daemon directory to start
the daemon (which will then start any missing engines).

\subsection{An Engine isn't running}
Check the process list to assert that the engine in question is really not
running (on UNIX, try ``ps -aux''). Look again. If the engine is
actually running but not responding via its HTTPD, remove the process.
Check the log file of the engine, generated in the engine
subdirectory, for any clues. If you are convinced that the engine is
not running, but find an ``archive\_active.lck'' lock file in the
engine directory, remove it. Now the daemon should be able to start
your engine.  

\subsection{I want to stop a Daemon}
Run stop-daemon.sh in the daemon directory, or check section
\ref{sec:daemon} for more on the daemon's HTTPD.

\subsection{A Daemon isn't running}
Run start-daemon.sh in the daemon directory. If the daemon keeps quitting,
check its log file for clues.

\subsection{Re-building a Master Index}
Whenever you add or remove a sub-archive, the master index in the
daemon directory could be obsolete: It might still list data in a
sub-archive that you removed, or it might not yet include the new
sub-archive. Another szenario: you suspect that the master index is
broken, because you can retrieve data from the individual sub-archive
but not via the master index. The recipe:
\begin{itemize}
\item Stop the daemon
\item Check, maybe rebuild indexconfig.xml, either manually or by
  using the helper script make\_indexconfig.pl
  (see section  \ref{sec:makeindexconfig}).
\item Start the daemon, which causes it to invoke the ArchiveIndexTool.
\end{itemize}

\section{Data Management}
The generation of daily sub-archives reduces the amount of data that
might be lost in case an ArchiveEngine crashes and cannot be restarted
by the ArchiveDaemon to one day. In the long run, however, it is
advisable to combine the daily sub-archives into bigger ones, for
example monthly. The smaller number of sub-archives is easier to
handle when it comes to backups. Is also provides slightly better
retrieval times. Depending on your situation, monthly archives might either
be too big to fit on a CD-ROM or ridiculously small, in which case you
should try weekly, bi-weekly, quarterly or other types of sub-archives.

In the following example, we assume that it's March 2004 and we want
to combine the two daily vacuum sub-archives from the previous section
into one for the month of February 2004:
\begin{lstlisting}[frame=none,keywordstyle=\sffamily]
cd vacuum/2004
mkdir 02_xx
ArchiveDataTool -copy 02_xx/index 02_19/index \
    -e "02/20/2004 02:00:00"
ArchiveDataTool -copy 02_xx/index 02_20/index \
    -s "02/20/2004 02:00:00" -e "02/21/2004 02:00:00"
\end{lstlisting}
\noindent Note that we assume a daily restart at 02:00 and thus we
force the ArchiveDataTool to only copy values from the time range
where we expect the sub-archives to have data. This practice somewhat
helps us to remove samples with wrong time stamps that result from
Channel Access servers with ill-configured clocks.

There is a perl command \INDEX{make\_compress\_script.pl} that aids in the
creation of a shell script for the ArchiveDataTool, but you need to
review it carefully before invokation.
After successfully combining the daily sub-archives for February 2004
into a monthly 2004/02\_xx, we need to
\begin{enumerate}
\item Stop the ArchiveDaemon because we are about to edit
      indexconfig.xml.
      The ArchiveEngines controlled by the daemon can run on.
\item Edit indexconfig.xml that listed the daily sub-archives for
      Feb.\ 2004 and replace them with the single 2004/02\_xx/index.
\item Remove or rename the master index file and re-create it with the new
      indexconfig.xml. This step is required because the ArchiveIndexTool
      will only add new data blocks to the master index, it will not
      remove existing ones. Since we no longer want to refer to the
      daily sub-archives, we need to recreate the master index.
\item Start the ArchiveDaemon again, check its online status.
\item One may now move the daily sub-archives that are no longer
      required to some temporary location. A month later, when we are
      convinced that nobody is still trying to use them, we can delete
      them.
\end{enumerate}

% ======================================================================
\section{"Serving" Computer}
% ======================================================================
\begin{figure}[htb]
\begin{center}
\InsertImage{width=0.9\textwidth}{archiveconfig_serve}
\end{center}
\caption{\label{fig:acServe}Tools used on a serving computer, refer to text.}
\end{figure}


\subsection{update\_server.pl} \label{sec:updateServer}.

\subsection{update\_indices.pl} \label{sec:updateIndices}.
