\chapter{Example Setup} \label{ch:examplesetup}
The following describes how the archiver toolset is used at the
\INDEX{Spallation Neutron Source (SNS)}. Other sites might need to
modify the tools in the ChannelArchiver/ExampleSetup directory for their
need.
For the SNS, all the archives are rooted in a directory ``/arch'' on
the computer ics-srv-archive1.
In there, a file ``\INDEX{archiveconfig.csv}'' describes the top-level layout:
 
\begin{lstlisting}[frame=none,keywordstyle=\sffamily]
# Archive configuration file
# This is TAB delimited!
# Careful when editing, do not disturb the TABs,
# because this file fails to parse unless
# everything is perfect.

# Type Name Port Desc       Time     Frequency
DAEMON ICS  4090 ICS Daemon
ENGINE tim  4091 Timing     08:00    daily
ENGINE mps  4092 Mach.Prot. 08:30    daily

DAEMON RF   4900 RF
ENGINE llrf 4901 LowLvl.    We 10:20 weekly
ENGINE hprf 4902 HiPwr.     We 10:30 weekly
\end{lstlisting}

\noindent This reads as follows:
We intend to run one ``ArchiveDaemon'' for the Integrated Control
System (ICS) and one for the channels related to Radio Frequency (RF).
More on the ArchiveDaemon follows, for now let it suffice that a
daemon starts archive engines and monitors their operation.
The ICS daemon should maintain one ArchiveEngine for the timing system
(tim) and one for the machine protection system (mps),
while the RF daemon has one engine for the low-level and one for the
high power RF (llrf, hprf).

This separation is somewhat arbitrary. We could have made ``llrf'' and
``hprf'' channel groups under one and the same engine. In fact all the above
could reside within one engine. It is, however, advisable to spread the channels
over different daemons and engines whenever different people deal with
the IOCs that host the channels, so that the engineers can
independently configure their archiving. 
In addition, you want to keep the amount of data collected by each
engine within certain bounds, for example: not more than one CD ROM
per month, one DVD per year, or whatever you plan to do for data
maintenance. Another reason is data safety: You can reduce the damage
caused by crashes of the engine by limiting the number of channels per engine.

\section{Current Archive Status}
The perl script ``\INDEX{make\_archive\_web.pl}'' creates a web page
displaying the current status of all the archive daemons and engines
that are listed in the file ``archiveconfig.csv''
(you could use other names for the main configuration file, but we
will stick with ``archiveconfig.csv'' in this manual).
For the SNS, this script it run periodically with the result being
redirected into the web server's document tree, so you can usually
reach the recent status from this web page:
\begin{lstlisting}[frame=none,keywordstyle=\sffamily]
    http://ics-srv-archive1/archive
\end{lstlisting}
\noindent and the result then looks somewhat like
Fig.~\ref{fig:archcfgstat}, showing you what engines are running, how
many channels are connected, and more.

\medskip

\begin{figure}[htb]
\begin{center}
\InsertImage{width=\textwidth}{archcfgstat}
\end{center}
\caption{\label{fig:archcfgstat}Example of the archive status web page
  generated by the make\_archive\_web.pl script.}
\end{figure}

\section{Directory Layout}
The perl script ``\INDEX{make\_archive\_dirs.pl}'' creates or updates a
directory tree based on the file ``archiveconfig.csv''. For the preceding
example, it would create subdirectories ``ICS'' and ``RF'' for the two
daemons, and the following engine directories:
\begin{lstlisting}[frame=none,keywordstyle=\sffamily]
  ICS/tim
  ICS/mps
  RF/llrf
  RF/hprf
\end{lstlisting}
\noindent Each daemon directory contains a daemon configuration file,
as well as start/stop scripts, all explained in the following section.
Each engine directory contains an ASCIIConfig subdirectory with an
example script that you might use to create an XML configuration file for the
ArchiveEngine, though the engineer responsible for the subsystem is
free to use any method of his/her choice as long as the result is a
configuration file for the engine that follows the naming convention
\begin{center}
{\it Daemon-Name} / {\it Engine-Name} / {\it Engine-Name}-group.xml~,\\
\end{center}
resulting in these for the prementioned example:
\begin{lstlisting}[frame=none,keywordstyle=\sffamily]
  ICS/tim/tim-group.xml
  ICS/mps/mps-group.xml
  RF/llrf/llrf-group.xml
  RF/hprf/hprf-group.xml
\end{lstlisting}
As long as you end up with engine configuration files of these names,
you can employ any text editor, a copy of the example script, or a
sophisticated toolset utilizing a relational database.

\section{ArchiveDaemon, Sub-Archives}
The ArchiveDaemon.pl perl script automates the operation of archive
engines. It requires a configuration file according to
ArchiveDaemon.dtd. In most cases, we use the configuration created by
make\_archive\_dirs.pl, which would look like this for the case of the
RF:
\begin{lstlisting}[frame=none,keywordstyle=\sffamily]
<?xml version="1.0" encoding="UTF-8" standalone="no"?>
<!DOCTYPE engines SYSTEM "/arch/ArchiveDaemon.dtd">
<engines>
  <engine>
    <desc>llrf</desc>
    <port>4901</port>
    <config>/arch/RF/llrf/llrf-group.xml</config>
    <weekly>We 10:20</weekly>
  </engine>
  <engine>
    <desc>hprf</desc>
    <port>4902</port>
    <config>/arch/RF/hprf/hprf-group.xml</config>
    <weekly>We 10:30</weekly>
  </engine>
</engines>
\end{lstlisting}

\noindent Based on this configuration, the daemon will 
\begin{enumerate}
\item Start an ArchiveEngine in ``RF/llrf'' that writes to
      a sub-archive named after the current day,
      e.g.\ ``RF/llrf/2004/02\_11/index''.
      Same for a second engine in ``RF/hprf''.
\item Periodically verify if engines that are supposed to run are
      actually running, attempting to start engines which are found missing.
\item Stop the ArchiveEngines each Wednesday at 10:20 respectively
      10:30 and restart them in new subdirectories
      (using the data of the restart for the path to the index file).  
\item Generate or update ``RF/indexconfig.xml'' whenever a new
      sub-archive is created for the vacuum or cooling data.
\item Periodically run ArchiveIndexTool on ``RF/indexconfig.xml'',
      generating or updating ``RF/master\_index''.
\item Provide a web page that lists the status of the two archive
      engines.
\end{enumerate}

\noindent As a result, we create separate sub-archives for the LLRF
and HPRF, a new one once per week, which provides some insurance
against crashes of an ArchiveEngine. (If you are paranoid, you can
choose daily sub-archives; compare the ICS setup from the beginning of
chapter \ref{ch:examplesetup}).
After running for a while, we will have created sub-archives like these:
\begin{lstlisting}[frame=none,keywordstyle=\sffamily]
  RF/llrf/2004/02_11/index
  RF/llrf/2004/02_18/index
  RF/llrf/2004/02_25/index
  ...
  RF/hprf/2004/02_11/index
  RF/hprf/2004/02_18/index
  RF/hprf/2004/02_25/index
  ...
\end{lstlisting}
\noindent In addition to each index file, there will also be
associated data files, but in most cases we identify an archive solely
by its index file.
For data retrieval, we can invoke e.g.\ ArchiveExport with the path to
any of the index files. This is, however, inconvenient because we will
only see data for one week of one subsystem at a time.
The periodic invocation of the ArchiveIndexTool allows us to view all
the RF data as a whole via the RF/master\_index.

\section{Master Indices}
The Index Tool allows creation of a master index
that covers more than one sub archive. For example, we can create
these two configuration files for the index tool, either manually or
with the help of make\_indexconfig.pl:
\begin{lstlisting}[frame=none,keywordstyle=\sffamily]
/data/vacuum/indexconfig.xml:
   Lists 2004/02_19/index and 2004/02_20/index
/data/cooling/indexconfig.xml:
   Lists 2004/02_19/index and 2004/02_20/index
\end{lstlisting}

\noindent After running ArchiveIndexTool in /data/vacuum and /data/cooling,
we will have two new indices. One refers to all the vacuum data, the
other to all the cooling data:
\begin{lstlisting}[frame=none,keywordstyle=\sffamily]
/data/vacuum/index
/data/cooling/index
\end{lstlisting}
\noindent Note that these are only index files. There are no new data
files because the new ``master'' index files will point to data blocks
in the existing data files, e.g. the one under /data/vacuum/2004/02\_19. 
It is also important to remember that the master index files include the
paths to the data files as instructed in the indexconfig.xml files.
According to the previous example,  
/data/vacuum/index was created from
/data/vacuum/indexconfig.xml which included the relative path
``2004/02\_19/index''. The vacuum master index will therefore point to
data files with a relative path like ``2004/02\_19/20040219''.
Whenever we use ``/data/vacuum/index'', the retrieval tools will prepend
the path to the index, ``/data/vacuum'', to the relative data 
file path found in the index, for example ``2004/02\_19/20040219'', and thus
find the data under its full path of
 e.g.\ ``/data/vacuum/2004/02\_19/20040219''.
We cannot move ``/data/vacuum/index'' to another location like
``/tmp/index''. The retrieval tools would then try to access
``/tmp/2004/02\_19/20040219'' and fail.

Having said that, it \emph{is} possible to generate master indices
that use the full, absolute paths to their data files by simply listing
the full paths to the sub-archives in indexconfig.xml. This is,
however, not recommended because it will increase the size of the
index files simply because the full path names are longer than the
relative paths. For the same reason it is advisable to use short path
names: When an index file points to many data blocks in many data
files, it makes quite some difference if you used a short-named
directory tree with paths like ``/data/vac/...'' as opposed to
``/user/data/channel-archiver/data/subsystems/vacuum-system/...''.

As a second step, we can further combine the master indices for vacuum
and cooling data into one index that covers all out data. By creating
``/data/indexconfig.xml'' in which we list ``vacuum/index'' and
``cooling/index'', and running the ArchiveIndexTool in
``/data'', we create ``/data/index'' which points to all our data.
Alternatively, we could have skipped the intermediate indices for
vacuum and cooling and created ``/data/indexconfig.xml'' from the
beginning like this:
\begin{lstlisting}[frame=none,keywordstyle=\sffamily]
/data/indexconfig.xml: Lists
   vacuum/2004/02_19/index
   vacuum/2004/02_20/index
   cooling/2004/02_19/index
   cooling/2004/02_20/index
\end{lstlisting}

\noindent In any case we end up with ``/data/index'' as an index for
all our vacuum and cooling data.

\section{Data Management}
The generation of daily sub-archives reduces the amount of data that
might be lost in case an ArchiveEngine crashes and cannot be restarted
by the ArchiveDaemon to one day. In the long run, however, it is
advisable to combine the daily sub-archives into bigger ones, for
example monthly. The smaller number of sub-archives is easier to
handle when it comes to backups. Is also provides slightly better
retrieval times. Depending on your situation, monthly archives might either
be too big to fit on a CD-ROM or ridiculously small, in which case you
should try weekly, bi-weekly, quarterly or other types of sub-archives.

In the following example, we assume that it's March 2004 and we want
to combine the two daily vacuum sub-archives from the previous section
into one for the month of February 2004:
\begin{lstlisting}[frame=none,keywordstyle=\sffamily]
cd /data/vacuum/2004
mkdir 02_xx
ArchiveDataTool -copy 02_xx/index 02_19/index \
    -e "02/20/2004 02:00:00"
ArchiveDataTool -copy 02_xx/index 02_20/index \
    -s "02/20/2004 02:00:00" -e "02/21/2004 02:00:00"
\end{lstlisting}
\noindent Note that we assume a daily restart at 02:00 and thus we
force the ArchiveDataTool to only copy values from the time range
where we expect the sub-archives to have data. This practice somewhat
helps us to remove samples with wrong time stamps that result from
Channel Access servers with ill-configured clocks.

At this time, there is no better tool nor a wrapper around the
ArchiveDataTool available, so a typical monthly data management run
will involve about 30 invocations of the ArchiveDataTool where one
needs to carefully adjust the sub archive paths, start times and end
times to suit the current month and year.

After successfully combining the daily sub-archives for February 2004
into a monthly 2004/02\_xx, we need to
\begin{enumerate}
\item Stop the ArchiveDaemon because we are about to edit
      indexconfig.xml.
      The ArchiveEngines controlled by the daemon can run on.
\item Edit /data/indexconfig.xml that listed the daily sub-archives for
      Feb.\ 2004 and replace them with the single 2004/02\_xx/index.
\item Remove or rename the master index file and re-create it with the new
      indexconfig.xml. This step is required because the ArchiveIndexTool
      will only add new data blocks to the master index, it will not
      remove existing ones. Since we no longer want to refer to the
      daily sub-archives, we need to recreate the master index.
\item Start the ArchiveDaemon again, check its online status.
\item One may now move the daily sub-archives that are no longer
      required to some temporary location. A month later, when we are
      convinced that nobody is still trying to use them, we can delete
      them.
\end{enumerate}

\noindent Again there is no tool available to automate the
indexconfig.xml update.
