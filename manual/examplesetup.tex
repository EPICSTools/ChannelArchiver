\chapter{Example Setup} \label{ch:examplesetup}
The following is meant as an example of how to use the various tools
together. Assume that we have channels related to vacuum readings and
channels related to a cooling system. As a result, we created the
following two configuration files for two different archive engines:
\begin{lstlisting}[frame=none,keywordstyle=\sffamily]
/data/vacuum/engineconfig.xml
/data/cooling/engineconfig.xml
\end{lstlisting}

\noindent Alternatively, we could have created only a single
configuration file, containing one ``Vacuum'' and one ``Cooling'' group of
channels. But subsystems are often assigned to different engineers, so
it is likely that we received these two configuration files from the
respective subsystem engineer and it is easiest to keep them separate.

\section{Engines and Sub-Archives}
By running ArchiveEngines in those two directories, we will get two
sub archives, that is: An index file and one or more data files will
be generated in /data/vacuum and /data/cooling.  In addition,
we might want to stop and restart the ArchiveEngines periodically,
let's say each day.  So even if one engine crashes, the worst that can
happen is that we loose data for one subsystem and one day. As a
result, we will now end up with multiple sub-archives. Their indices
could be this:
\begin{lstlisting}[frame=none,keywordstyle=\sffamily]
/data/vacuum/2004/02_19/index
/data/vacuum/2004/02_20/index
/data/cooling/2004/02_19/index
/data/cooling/2004/02_20/index
\end{lstlisting}

\noindent There will also be data files like
/data/vacuum/2004/02\_19/20040219, but for the most part we can
identify a sub-archive via its index file, so we listed only those.
For data retrieval, we can invoke ArchiveExport with the
path to any of the four index files. We can also list all four index
files in the configuration for our network data server. This is,
however, inconvenient because we will only see data for one day of one
subsystem at a time. 

\section{Master Indices}
The Index Tool allows creation of a master index
that covers more than one sub archive. For example, we can create
these two configuration files for the index tool, either manually or
with the help of make\_indexconfig.pl:
\begin{lstlisting}[frame=none,keywordstyle=\sffamily]
/data/vacuum/indexconfig.xml:
   Lists 2004/02_19/index and 2004/02_20/index
/data/cooling/indexconfig.xml:
   Lists 2004/02_19/index and 2004/02_20/index
\end{lstlisting}

\noindent After running ArchiveIndexTool in /data/vacuum and /data/cooling,
we will have two new indices. One refers to all the vacuum data, the
other to all the cooling data:
\begin{lstlisting}[frame=none,keywordstyle=\sffamily]
/data/vacuum/index
/data/cooling/index
\end{lstlisting}
\noindent Note that these are only index files. There are no new data
files because the new ``master'' index files will point to data blocks
in the existing data files, e.g. the one under /data/vacuum/2004/02\_19. 
It is also important to remember that the master index files include the
paths to the data files as instructed in the indexconfig.xml files.
According to the previous example,  
/data/vacuum/index was created from
/data/vacuum/indexconfig.xml which included the relative path
``2004/02\_19/index''. The vacuum master index will therefore point to
data files with a relative path like ``2004/02\_19/20040219''.
Whenever we use ``/data/vacuum/index'', the retrieval tools will prepend
the path to the index, ``/data/vacuum'', to the relative data 
file path found in the index, for example ``2004/02\_19/20040219'', and thus
find the data under its full path of
 e.g.\ ``/data/vacuum/2004/02\_19/20040219''.
We cannot move ``/data/vacuum/index'' to another location like
``/tmp/index''. The retrieval tools would then try to access
``/tmp/2004/02\_19/20040219'' and fail.

Having said that, it \emph{is} possible to generate master indices
that use the full, absolute paths to their data files by simply listing
the full paths to the sub-archives in indexconfig.xml. This is,
however, not recommended because it will increase the size of the
index files simply because the full path names are longer than the
relative paths. For the same reason it is advisable to use short path
names: When an index file points to many data blocks in many data
files, it makes quite some difference if you used a short-named
directory tree with paths like ``/data/vac/...'' as opposed to
``/user/data/channel-archiver/data/subsystems/vacuum-system/...''.

As a second step, we can further combine the master indices for vacuum
and cooling data into one index that covers all out data. By creating
``/data/indexconfig.xml'' in which we list ``vacuum/index'' and
``cooling/index'', and running the ArchiveIndexTool in
``/data'', we create ``/data/index'' which points to all our data.
Alternatively, we could have skipped the intermediate indices for
vacuum and cooling and created ``/data/indexconfig.xml'' from the
beginning like this:
\begin{lstlisting}[frame=none,keywordstyle=\sffamily]
/data/indexconfig.xml: Lists
   vacuum/2004/02_19/index
   vacuum/2004/02_20/index
   cooling/2004/02_19/index
   cooling/2004/02_20/index
\end{lstlisting}

\noindent In any case we end up with ``/data/index'' as an index for
all our vacuum and cooling data.

\section{Automatization}
The ArchiveDaemon program will automate most of what we described so
far. By creating a file ``/data/ArchiveDaemon.xml'' as follows and
running the archive daemon in ``/data'', the daemon will 
\begin{enumerate}
\item Start an ArchiveEngine in ``/data/vacuum'' that writes to
      a daily sub-archive like ``/data/vacuum/2004/02\_19/index''.
      Same for a second engine in ``/data/cooling'' ...
\item Stop each ArchiveEngine at 02:00~AM and restart it in a new
      subdirectory.  
\item Generate or update ``/data/indexconfig.xml'' whenever a new
      sub-archive is created for the vacuum or cooling data.
\item Periodically run ArchiveIndexTool on ``/data/indexconfig.xml'',
      generating or updating ``/data/master\_index''.
\item Provide a web page that lists the status of the two archive
      engines.
\end{enumerate}

\noindent This is the example ArchiveDaemon.xml:
\begin{lstlisting}[frame=none,keywordstyle=\sffamily]
<?xml version="1.0" encoding="UTF-8" standalone="no"?>
<!DOCTYPE engines SYSTEM "ArchiveDaemon.dtd">
<engines>
  <engine>
   <desc>Vacuum</desc>          <port>4813</port>
   <config>/data/vacuum/engineconfig.xml</config>
   <daily>02:00</daily>
  </engine>
  <engine>
   <desc>Cooling</desc>         <port>4814</port>
   <config>/data/cooling/engineconfig.xml</config>
   <daily>02:00</daily>
  </engine>
</engines>
\end{lstlisting}

\section{Data Management}
The generation of daily sub-archives reduces the amount of data that
might be lost in case an ArchiveEngine crashes and cannot be restarted
by the ArchiveDaemon to one day. In the long run, however, it is
advisable to combine the daily sub-archives into bigger ones, for
example monthly. The smaller number of sub-archives is easier to
handle when it comes to backups. Is also provides slightly better
retrieval times. Depending on your situation, monthly archives might either
be too big to fit on a CD-ROM or ridiculously small, in which case you
should try weekly, bi-weekly, quarterly or other types of sub-archives.

In the following example, we assume that it's March 2004 and we want
to combine the two daily vacuum sub-archives from the previous section
into one for the month of February 2004:
\begin{lstlisting}[frame=none,keywordstyle=\sffamily]
cd /data/vacuum/2004
mkdir 02_xx
ArchiveDataTool -copy 02_xx/index 02_19/index \
    -e "02/20/2004 02:00:00"
ArchiveDataTool -copy 02_xx/index 02_20/index \
    -s "02/20/2004 02:00:00" -e "02/21/2004 02:00:00"
\end{lstlisting}
\noindent Note that we assume a daily restart at 02:00 and thus we
force the ArchiveDataTool to only copy values from the time range
where we expect the sub-archives to have data. This practice somewhat
helps us to remove samples with wrong time stamps that result from
Channel Access servers with ill-configured clocks.

At this time, there is no better tool nor a wrapper around the
ArchiveDataTool available, so a typical monthly data management run
will involve about 30 invocations of the ArchiveDataTool where one
needs to carefully adjust the sub archive paths, start times and end
times to suit the current month and year.

After successfully combining the daily sub-archives for February 2004
into a monthly 2004/02\_xx, we need to
\begin{enumerate}
\item Stop the ArchiveDaemon because we are about to edit
      indexconfig.xml.
      The ArchiveEngines controlled by the daemon can run on.
\item Edit /data/indexconfig.xml that listed the daily sub-archives for
      Feb.\ 2004 and replace them with the single 2004/02\_xx/index.
\item Remove or rename the master index file and re-create it with the new
      indexconfig.xml. This step is required because the ArchiveIndexTool
      will only add new data blocks to the master index, it will not
      remove existing ones. Since we no longer want to refer to the
      daily sub-archives, we need to recreate the master index.
\item Start the ArchiveDaemon again, check its online status.
\item One may now move the daily sub-archives that are no longer
      required to some temporary location. A month later, when we are
      convinced that nobody is still trying to use them, we can delete
      them.
\end{enumerate}

\noindent Again there is no tool available to automate the
indexconfig.xml update.