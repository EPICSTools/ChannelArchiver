% ======================================================================
\chapter{Example Setup} \label{ch:examplesetup}
% ======================================================================
The following describes how the archiver toolset is used at the
\INDEX{Spallation Neutron Source (SNS)}. One can of course configure
individual engines, start and stop them manually, and do the same with
index tools and network data servers. You \emph{should} in fact
initially do exactly that in order to become familiar with the pieces
of the toolset.
Ultimately, however, we try to base as much as possible on a central
configuration, and automate the rest via the scripts in the
ChannelArchiver/ExampleSetup directory.
We distinguish between two types of computers:
\begin{itemize}
\item Sampling Machine:\\
      A computer that runs ArchiveEngine instances.
\item Serving Machine:\\
      A computer that uses the ArchiveIndexTool to create
      additional binary indices and runs the ArchiveDataServer.
\end{itemize}

\noindent There might be more than one 'sampling' computer as well as more
than one 'serving' computer.  A single machine might perform both
functions, but in general they are different, networked computers, and
consequently tools are required to make the data collected on the
``sampling'' computer available on the ``server''.  One could use NFS,
but we prefer secure copy (scp) in order to decouple the computers as
best as possible.

We want to be able to move an engine from one computer to another, and
still keep an overview.  Therefore a file ``/arch/archiveconfig.xml''
describes the complete archive setup: Which engines run where, and how
the data gets served.  On some computers, for example
ics-srv-archive1, further subdirectories of ``/arch'' are used to run
engines.  On another computer, for example ics-srv-web2,
subdirectories contain data copied from archive1 so that the data
server can serve it.

% ======================================================================
\section{Setup, archiveconfig.xml}
% ======================================================================
Each computer needs to have the same copy of
``/arch/\INDEX{archiveconfig.xml}'',
and the scripts from the ``ExampleSetup'' directory of the archiver
sources need to be copied into ``\INDEX{/arch/scripts}''.
You might generate and distribute ``archiveconfig.xml'' manually or use a
relational database.  People who have used previous releases of the
archive toolset might remember the ``archiveconfig.csv'' file. There is a
tool ``scripts/\INDEX{convert\_archiveconfig\_to\_xml.pl}'' to convert that
file into an ``archiveconfig.xml'' skeleton.

\lstinputlisting[float=htb,keywordstyle=\sffamily,caption={archiveconfig.xml},label=lst:archiveconfig]{../ExampleSetup/archiveconfig.xml}

\clearpage

\noindent The ``archiveconfig.xml'' file describes the complete archive layout,
using the following elements:
\begin{itemize}
\item \INDEX{root tag}: Names the root directory, typically ``\INDEX{/arch}''.
      This has to be the same directory name on all computers,
      since they all use the same ``archiveconfig.xml''.
\item \INDEX{serverconfig tag}: Location of the server configuration to be
      created. See section update\_server.pl, \ref{sec:updateServer}.
\item \INDEX{mailbox tag}: Used to communicate from the ``sampling'' to the
      ``serving'' machines.
\item \INDEX{daemon tag}: Configures an archive daemon and its engines,
      described in \ref{sec:exampleSample},
      as well as how that data should be indexed and served,
      see \ref{sec:exampleServe}.
\end{itemize}

% ======================================================================
\section{Sampling Computer} \label{sec:exampleSample}
% ======================================================================
\begin{figure}[htb]
\begin{center}
\InsertImage{width=0.9\textwidth}{archiveconfig_sample}
\end{center}
\caption{\label{fig:acSample}Tools used on a sampling computer, refer to text.}
\end{figure}

\noindent Instead of running just one archive engine, it is often desirable to
run several. This way, there can be separate engines for each
subsystem, possibly maintained by different people.  In addition,
periodic restarts, for example once a week, create a separate
\INDEX{sub-archive} with each restart, thereby limiting the
possibility for data loss in case an engine crashes or creates corrupt
archives.

For each subsystem, an ``ArchiveDaemon'' program manages one or more
engines, monitors their condition, and performs the periodic restarts.
For example, we might run one `daemon' for the Integrated Control
System (ICS) and one for the channels related to Radio Frequency (RF).
The ICS daemon should maintain one ArchiveEngine for the timing system
(tim) and one for the machine protection system (mps), while the RF
daemon has one engine for the low-level and one for the high power RF
(llrf, hprf).

This separation is somewhat arbitrary. We could have made ``llrf'' and
``hprf'' channel groups under one and the same engine. In fact all the
above could reside within one engine, and the result would probably be
less CPU load compared to the setup with multiple engines.  It is,
however, advisable to spread the channels over different daemons and
engines whenever different people deal with the IOCs that host the
channels, so that the engineers can independently configure their
archiving.  In addition, you want to keep the amount of data collected
by each engine within certain bounds, for example: not more than one
CD ROM per month, one DVD per year, or whatever you plan to do for
data maintenance. You can of course also follow the approach that most
sites use in reality: Wait until all disks are full, then panic.
In which case, another reason is data safety: You can reduce the
damage caused by crashes of one engine to a certain number of channels
and the data for the restart period, for example one week.

% ======================================================================
\subsection{Configuration}
% ======================================================================
The configuration of the sampling computer is primarily done
in the $<$daemon$>$ sections of archiveconfig.xml, described in
section \ref{sec:daemon}.

% ======================================================================
\subsection{update\_archive\_tree.pl}
% ======================================================================
Initially and after every change to the configuration,
the \INDEX{update\_archive\_tree} script is used to create the necessary
infrastructure.
This script reads archiveconfig.xml and create all the subdirectories,
daemon config files, and skeleton engine configurations which are
meant to run on the local computer.
Run it with ``-h'' to see available options.

\NOTE In order to determine what daemons should run on the
\INDEX{local host},
the \INDEX{host name} in the ``\INDEX{run tags}'' is used as a
\INDEX{regular expression} for the host name.
So when specifying that a certain daemon should run on ``archive1'':
\begin{lstlisting}[keywordstyle=\sffamily]
 <daemon ...
    <run>archive1</run>
    ...
\end{lstlisting}
... that daemon will run on computers called ``ics-srv-archive1'' as well
as ``archive1.sns.ornl.gov'' etc.
The same applies to the host tags which specify where a
data server should run.

For the example from listing \ref{lst:archiveconfig}, we would get
these subdirectories:
\begin{lstlisting}[frame=none,keywordstyle=\sffamily]
  RF
  RF/llrf
  RF/llrf/ASCIIConfig
\end{lstlisting}
\noindent Each directory contains configuration files and scripts
explained in the following. Each engine directory also contains an
``ASCIIConfig'' subdirectory with a script ``convert\_example.sh''
that you might use to create the XML configuration file for the
ArchiveEngine from ASCII configuration files, though the engineer
responsible for the subsystem is free to use any method of his/her
choice as long as the result is a configuration file for the engine
that follows the naming convention
\begin{center}
\emph{Daemon-Directory} / \emph{Engine-Dir.} / \emph{Engine-Dir.}-group.xml,\\
\end{center}
If you want to use the ASCIIConfig directory, check section
\ref{sec:ASCIIConfig} for the format of the ASCII configuration files.

% ======================================================================
\subsection{ArchiveDaemon} \label{sec:daemon}
% ======================================================================
\begin{figure}[htb]
\begin{center}
\InsertImage{width=0.75\textwidth}{daemon}
\end{center}
\caption{\label{fig:daemon}Archive Daemon, refer to text.}
\end{figure}

\noindent The \INDEX{ArchiveDaemon} is a script that automatically starts,
monitors and restarts ArchiveEngines on the local host. It is based on
ideas by Thomas Birke, who implemented a similar \INDEX{CAManager}
tool while at LANL. The daemon includes a built-in web server, so by
listing several ArchiveEngines that are meant to run on a host in the
ArchiveDaemon's configuration file, one can check the status of all
these engines on a single web page as shown in Fig.~\ref{fig:daemon}.

The daemon will attempt to start any ArchiveEngine
that it does not find running. In addition, the daemon can
periodically stop and restart ArchiveEngines in order to create
e.g.\ daily sub-archives.  Furthermore, it adds information about
each completed sub-archive to a mailbox directory,
allowing the indexing mechanism to create the necessary indices
and update the data server configuration.

Before using the ArchiveDaemon, one should be familiar
with the configuration of a single ArchiveEngine (sec.\ \ref{sec:engine}),
and how to start and stop it manually.

\lstinputlisting[float=htb,keywordstyle=\sffamily,caption={Example
Archive Daemon configuration for listing \ref{lst:archiveconfig}.},label=lst:daemonconfigex]{../ExampleSetup/ArchiveDaemon.xml}

% ======================================================================
\subsubsection{Configuration}
% ======================================================================
You will typically \emph{not} directly configure an archive daemon,
but instead specify its configuration in the ``archiveconfig.xml''
file. The ``update\_archive\_tree.pl'' script will then generate
those daemon configurations needed for the local computer, and ignore
those meant to run on other machines.
\clearpage

In principle, you can also use the archive daemon by manually creating
its configuration, just like you can manually start and stop engines
without using the daemon. But in here we assume that you start with a
configuration as shown in listing \ref{lst:archiveconfig},
which calls for an ``RF'' daemon to maintain a ``llrf'' engine on the host
``archive1''. So on ``archive1'', a subdirectory ``/arch/RF'' will be
created with the daemon configuration shown in listing 
\ref{lst:daemonconfigex}, which complies with the DTD from listing
\ref{lst:daemonconfigdtd}.

The following is an explanation of the daemon-related tags in the
``archiveconfig.xml'' file. Compare with the created daemon
configuration to see how they get extracted from the global
configuration file.

\begin{itemize}
\item \INDEX{run tag}:\\
      Both the daemon and each engine have this element, which
      specifies the host name where a daemon and engines should run.
      The names are regular expressions.
      In principle, one could think about a setup where
      RF/llrf runs on ``archive1'' and RF/hprf runs on
      ``archive2'', and consequently both machines run an
      ``RF'' daemon:
\begin{lstlisting}[keywordstyle=\sffamily]
 <daemon directory='RF'>
    <run>archive[12]</run>
    ...
    <engine directory='llrf'> 
       <run>archive1</run>
       ...
    <engine directory='hprf'> 
       <run>archive2</run>
\end{lstlisting}
      This has not been tested. Typically, a daemon and its engines
      are all on the same computer.

      The generated daemon config file only includes the configuration
      needed for the local computer.  So instead of duplicating
      $<$run$>$localhost$<$/run$>$, it is omitted in the daemon config file.
\item \INDEX{desc tag}:\\
      A short description.
\item \INDEX{port tag}:\\
      Both the daemon and each engine have this mandatory element, which
      determines the port number of the HTTP server. Section
      \ref{sec:daemonserver} describes the HTTP server of the daemon, while
      section \ref{sec:engineport} explains the engine HTTPD.

      \NOTE The port numbers used by the Archive Daemons and all the Archive
      Engines need to be different. You cannot use the same port number more
      than once per computer.
\item \INDEX{mailbox tag}:\\
      Directory that the daemon uses to communicate with the data server.
\item \INDEX{engine tag}:\\
      Specifies the sub-directory for each engine under a daemon.
\item \INDEX{config tag}:\\
      This element of each engine entry contains the path to the
      configuration file of the respective ArchiveEngine, see section
      \ref{sec:engineconfig}. The update\_archive\_-tree.pl script
      creates this entry based on the engine directory from
      archiveconfig.xml as ``{\it engine-}-group.xml''. You can only
      influence it if you choose to not use archiveconfig.xml and
      instead create the daemon configurations manually.
\item \INDEX{restart tag}:\\
      When provided, it specifies when the daemon should re-start an engine.
      \begin{itemize}
      \item \INDEX{daily tag}:\\
         The element must contain a time in the
	 format ``HH:MM'' with 24-hour HH and minutes MM. One example
	 would be ``02:00'' for a restart at 2~am each morning.
      \item \INDEX{weekly tag}:\\
	 Weekly is similar to daily, but using an element that contains the day
	 of the week (Mo, Tu,  We, Th, Fr, Sa, Su) in addition to the time
	 on that day in 24-hour format, e.g.\ ``We 08:00''. In this example,
	 the daemon will attempt a restart every Wednesday,
         8'o clock in the morning.
      \item \INDEX{timed tag}:\\
	 In this case, the element needs to contain a start/duration time pair
	 in the format ``HH:MM/HH:MM''. The first, pre-slash 24-hour time stamp
	 indicates the start time, and the second 24-hour time, trailing the
	 slash, specifies the runtime. The engine will be launched at the
	 requested start time and run for the duration of the runtime. As an
	 example, ``08:00/01:00'' requests that the daemon starts the engine at
	 08:00 and stops it after one hour, probably around 09:00,
         each day.
      \item \INDEX{hourly tag}:\\
         The element must contain a number specifying hours: A value
	 of 2.0 will cause a restart every 2 hours. The hourly restart
	 is quite inefficient and primarily meant for testing.
      \end{itemize}
      \NOTE It is advisable to stagger the restart times of your engines
      such that they don't all restart at the same day and time in order to
      reduce the CPU and network load for the ChannelAccess
      re-connects.
      See also show\_restarts.pl in section \ref{sec:example:statustools}.
\item \INDEX{dataserver tag}:\\
      Specifies the name of the data server host.
\end{itemize}

\lstinputlisting[float=htb,keywordstyle=\sffamily,caption={XML DTD for the Archive Daemon Configuration},label=lst:daemonconfigdtd]{../ExampleSetup/ArchiveDaemon.dtd}

% ======================================================================
\subsubsection{Starting, running, stopping daemons and engines}
% ======================================================================
The following scripts, some created by ``update\_archive\_tree.pl'' in
each daemon and engine subdirectory, are used to control the daemons
and engines:
\begin{itemize}
\item scripts/\INDEX{start\_daemons.pl} \\
      Starts all daemons meant to run on the local computer.
\item scripts/\INDEX{stop\_daemons.pl} \\
      Stops all daemons. See ``-h'' for available options, which
      includes ``-p'' to also stop all engines.
\item daemon-dir/\INDEX{run-daemon.sh} \\
      Starts the daemon for this subsystem.
      The daemon will then start all engines which are not already
      found running.
\item daemon-dir/\INDEX{stop-daemon.sh} \\
      Stops the daemon for this subsystem.

      \NOTE This does not stop the engines, only the daemon.
\item daemon-dir/\INDEX{view-daemon.sh} \\
      Script to run '\INDEX{lynx}', the text-based web browser, with
      the URL of the daemon HTTPD.
\item daemon-dir/engine-dir/\INDEX{stop\_engine.sh} \\
      Stop the engine.

      \NOTE There is no script to start an individual engine, because the
      daemon will eventually start any engine that's not running.
      So stopping an engine is just another means of triggering a
      restart.
\end{itemize}

\noindent Each engine subdirectory contains
\begin{itemize}
\item \INDEX{archive\_active.lck}\\
  Lock file of the ArchiveEngine, exists while an engine is running.
\item YYYY/MM\_DD/index \\
  A subdirectory for index and data files of the sub-archive.  If the
  ArchiveDaemon is configured to perform daily restarts, the format
  uses the year, month and day to build the path name.
\item \INDEX{current\_index} \\
  A soft-link to the currently used index.
\end{itemize}

\noindent In addition, there are log files like ArchiveDaemon.log and
ArchiveEngine.log created in the daemon and engine directories which
log the diagnostics output of the respective programs.

% ======================================================================
\subsubsection{\INDEX{Daemon Web Server}} \label{sec:daemonserver}
% ======================================================================
You can use any web browser to view the daemon's web pages,
which look similar to Fig.~\ref{fig:daemon}.
The URL follows the format
\begin{lstlisting}[keywordstyle=\sffamily]
    http://host:port
\end{lstlisting}
\noindent where ``host'' is the name of the computer where the
ArchiveDaemon is running, and ``Port'' is the TCP port that was
specified in the config file.

\NOTE Instead of using Mozilla, Firefox, or any of the other
full-blown graphical web browsers, it is often more practical to use
``\INDEX{lynx}''. This text-mode web browser shows the same
information, but starts up quicker and uses less CPU
and memory resources, which can then be used by archive engines and the
operating system's disk cache.

The main daemon web page lists all the archive engines that this
daemon controls with their status. The first column also contains
links to the individual archive engines. The status shows any of the
following:

\begin{itemize}
\item ``N/M channels connected''\\
      This means the ArchiveEngine is running and responding,
      telling us that N out of a total of M channels have connected.
      If not all channels could connect, you might want to follow
      the link to the individual engine and further down to its
      channel groups and channels, and determine what channels are
      missing and why: Is an IOC down on purpose? Is an IOC
      disconnected because of network problems? Does a channel simply
      not exist, i.e.\ the engine's configuration is wrong?
\item ``Not Running''\\
      This means that the respective ArchiveEngine did not respond
      when we queried it, and there is no ``archive\_active.lck'' lock
      file. This combination usually means that the engine is really
      not running (except for the Note below about startup).

      The first step in debugging would be to check the engine's
      directory for a log file. Does it indicate why the engine could
      not start? Then check the daemon's log file. It should list the
      exact command used to start the engine. You can try that
      manually to check why it didn't work.
\item ``Disabled.''\\
      The web interface of the daemon contains a link for each engine
      that disables the engine. This places a file
      ``\INDEX{DISABLED.txt}'' in the engine directory and stops the
      engine.  As you might have guessed, the daemon will not attempt
      to start engines as long as the ``DISABLED'' file is found. This
      is a convenient way to temporarily disable an engine without
      removing it from the daemon's configuration.
\item ``Unknown. Found lock file''\\
      This means that the respective ArchiveEngine did not respond
      when we queried it, but there is an ``archive\_active.lck'' lock
      file. This could have two reasons. It could mean that the engine
      is running but it was temporarily unable to respond to the
      daemon's request. An example would be that the engine is really
      busy writing and dealing with ChannelAccess, so that its web
      server had to wait and the daemon timed out. All should be fine
      again after some time.

      If, on the other hand, the situation persists, it usually means
      that the engine is hung or has crashed, so that it does not
      respond and the lock file was left behind.
      See Crashes on page \pageref{sec:crash}.
\end{itemize}

\NOTE The daemon queries the engines only every once in a while and
leaves them undisturbed most of the time.
Especially after startup, all engines will show up as ``Not Running''
in the daemon's web page while in fact most of them are already
running. Then you will see many disconnected channels while the
engines did in fact already connect to all channels. 
If you are impatient, you can click on the links to the individual
engines to get a more up-to-date snapshot of each engine's status.

Instead of using the scripts to stop and ArchiveDaemon, one can
directly access the ``/stop'' URL of the daemon's HTTPD,
e.g. ``http://localhost:4610/stop''.  Similar to the ArchiveEngine's
HTTPD, this URL is not accessible by following links on the HTTPD's
web pages. You will have to type the URL. This is meant to prevent web
robots or a monkey who is sitting in front of the computer and
clicking on every link from accidentally stopping the daemon, although
it has never been tested with an actual monkey.

Finally, the daemon will respond to the URL ``/\INDEX{postal}'' by stopping
every ArchiveEngine controlled by the daemon, followed by stopping
itself. ``Postal'' is an abbreviation for ``POstpone STopping the
daemon until ALl engines quit''. It is not at all related to ``going
postal'', and to our best knowledge no \INDEX{USPS} employees have been hurt
during the development of this software.

% ======================================================================
\subsection{Status Information} \label{sec:example:statustools}
% ======================================================================
The following scripts use the HTTPD of the daemon and engine
as well as generic Unix tools to create status overviews.
Some can be used in ``\INDEX{cron}'' jobs to periodically update
web pages or send regular status emails.

\begin{itemize}
\item \INDEX{make\_archive\_infofile.pl} \\
      Creates summary of daemon and engine status,
      what is running and what is missing,
      how many channels connected etc.
      See ``-h'' option for details.
\item \INDEX{make\_archive\_web.pl} \\
      Similar, but creating a web page. Fig.~\ref{fig:archcfgstat}
      shows one example.
\item \INDEX{engine\_write\_durations.pl} \\
      Summary of engine performance: How many channels, average values
      per seconds.
\item \INDEX{show\_restarts.pl} \\
      Creates an overview of restart times, meant as an aid for selecting
      restart times that either coincide or avoid other restarts based
      on your preference. 
\item \INDEX{show\_engines.pl} \\
      Parses output of ``ps'' command for engine related information.
\item \INDEX{show\_sizes.pl} \\
      Parses output of ``du'' for size of sub-archives.
\end{itemize}

\begin{figure}[htb]
\begin{center}
\InsertImage{width=\textwidth}{archcfgstat}
\end{center}
\caption{\label{fig:archcfgstat}Example of the archive status web page
  generated by the make\_archive\_web.pl script.}
\end{figure}

% ======================================================================
\section{\INDEX{Sub-Archives}}
% ======================================================================
Based on the example configuration used in this chapter,
the daemon will 
\begin{enumerate}
\item Start an ArchiveEngine in ``RF/llrf'' that writes to
      a sub-archive named after the current day,
      e.g.\ ``RF/llrf/2004/02\_11/index''.
\item Periodically verify if engines that are supposed to run are
      actually running, attempting to start engines which are found missing.
\item Stop the llrf engine each Wednesday at 10:20, and restart it in
      a new subdirectory, using the date of the restart for the path
      name to the new index file.
\item Maintain a ``\INDEX{current\_index}'' soft link to point to the current
      sub-archive.
\item Generate a file in the mailbox directory with information about
      the ``old'' sub-archive, the one generated by the engine that was
      just stopped.
\end{enumerate}

\noindent As a result, we create weekly sub-archives for the LLRF like
these on the sampling computer ``archive1'':
\begin{lstlisting}[frame=none,keywordstyle=\sffamily]
  ...
  /arch/RF/llrf/2006/03_08/index
  /arch/RF/llrf/2006/03_15/index
  /arch/RF/llrf/2006/03_22/index
  /arch/RF/llrf/current_index  -> 2006/03_22/index
\end{lstlisting}
\noindent In addition to each index file, there will of course also be
associated data files, but for retrieval purposes we identify an
archive solely by its index file: We can invoke e.g.\ ArchiveExport
with the path to any of the index files. 
Unfortunately, whichever index we choose, we will only see data for
one week of that one subsystem at a time.
The ``serving'' computer will therefore use additional index files.

% ======================================================================
\section{Serving Computer}  \label{sec:exampleServe}
% ======================================================================
\begin{figure}[htb]
\begin{center}
\InsertImage{width=0.9\textwidth}{archiveconfig_serve}
\end{center}
\caption{\label{fig:acServe}Tools used on a serving computer, refer to text.}
\end{figure}

\noindent The serving computer might actually be the same as the
sampling machine, or it can be a different computer. In any case, all
computers must have access to the same archiveconfig.xml file and
mailbox directory.
The following describes the scenario used at the SNS with 
different computers and no writable NFS share between the
affected machines.
The idea is to copy the sub-archives over to the serving computer, and
create further indices. For the data example from the
previous section, we will get the following on the serving computer ``web2'':
\begin{lstlisting}[frame=none,keywordstyle=\sffamily]
  ...
  /arch/RF/llrf/2006/03_08/index
  /arch/RF/llrf/2006/03_15/index
  /arch/RF/llrf/master_index     
  /arch/RF/llrf/current_index  -> /archive1/arch/RF/llrf/current_index
\end{lstlisting}
\noindent All sub-archives which are closed, i.e.\ no longer updated
by a running engine, have been copied from the sampling computer,
where they could now be deleted.
A binary ``master\_index'' is maintained for all the copied data.
Depending on the configuration, this could also have been a list
index.

The daemon on the sampling computer provides a
``current\_index'' soft link to the current sub-archive index.
At the SNS, the serving computer ``web2'' is a separate machine with
a read-only NFS mount ``/archive1'' to ``archive1:/arch''.
A manually created soft link from /arch/RF... to the NFS share allows
access to the current index in the familiar place, even though read
access will of course follow the first soft link to the
``current\_index'' on the NFS share,
then on the actual current index.

% ======================================================================
\subsection{Configuration}
% ======================================================================
Of primary interest to the serving computer are the $<$dataserver$>$
sections of archiveconfig.xml, which are allowed for each daemon and
engine entry:

\begin{lstlisting}[frame=none,keywordstyle=\sffamily]
    <dataserver>
      <current_index key='4901'>llrf</current_index>
      <index type='binary' key='4902'>LLRF data</index>
      <host>web2</host>
    </dataserver>
\end{lstlisting}

\begin{itemize}
\item \INDEX{current\_index tag}:\\
      This option is only allowed for engine data server entries.
      When provided, the ``\INDEX{current\_index}'' link created by the
      archive daemon for this engine will be served by the
      data server as its own key and name.

      This of course requires the ``serving'' computer to have access
      to the current\_index, which resides on the ``sampling''
      machine. At the SNS, we have a read-only NFS mount from archive1
      onto web2 for this very purpose.
      ``web2:/arch/RF/llrf/current\_index'' for example is a manually
      created soft-link to the actual current\_index soft-link on the
      NFS share.
\item \INDEX{index tag}: \\
      Configures the type of index to create as ``list'' or ``binary''.
      For an engine, the index covers every data found under the engine
      directory, typically combining the sub-archives from daily or weekly
      restarts into one master\_index.

      For a daemon, the index combines the master\_indices of engines
      under the daemon into one daemon-level master\_index. This
      requires that there are such engine indices, which need to be
      configured for each engine entry.

      If a key is provided, the index is added to the data server
      configuration. It might make sense to only provide a key for the
      daemon-level index, while the engine-level indices are only used
      as intermediate steps in constructing the daemon index.            
\item \INDEX{host tag}: \\
      The host where the data server should run. Dataserver entries where
      the host element, used as a regular expression, does not match the
      local hostname, are ignored.
\end{itemize}

\noindent In addition to the indices specified in $<$dataserver$>$
sections of the archive configuration, two more will
be created by the update\_indices.pl script.

\subsection{\INDEX{send\_mailbox.pl}}
In the absence of a writable NFS share for the mailbox directory,
this script is run by a \INDEX{cron}-job to send the contents of the
\INDEX{mailbox} directory to other computers.
For now, this means: It is periodically invoked on ``sampling''
computers to send the current mailbox content to the ``serving''
machine. See command-line help for details.

\subsection{\INDEX{update\_server.pl}} \label{sec:updateServer}
This script should be invoked periodically by a cron-job.
It checks the mailbox directory for information about new data on a
``sampling'' computer, compares with the archiveconfig.xml information
to decide which of that data should be served on this machine.

It uses \INDEX{secure-copy (scp)} to pull new sub-archives onto this computer,
optionally with \INDEX{md5 checksum}, and finally invokes update\_indices.pl.

\subsection{\INDEX{update\_indices.pl}} \label{sec:updateIndices}
Reads archiveconfig.xml and creates an indexconfig.xml for each
engine and daemon directory that has a $<$dataserver$>$...$<$host$>$
entry for the local machine.
Where binary indices were requested, the ArchiveIndexTool is invoked.

Next, list indices ``current.xml'' and ``all.xml'' are created, 
containing all the ``current\_index'' entries from the configuration
respectively the remaining index entries.
Finally, the serverconfig.xml is updated, starting with entries for
``all.xml'' (key1) and ``current.xml'' (key 2), followed by the
indices listed in archiveconfig.xml.

% ======================================================================
\section{Common Tasks}
% ======================================================================
\subsection{Modify Engine's Request Files}
Locate your archive engine directory and the engine configuration
file, for example /arch/RF/llrf/llrf-group.xml for the ``llrf'' engine
maintained by the ``RF'' daemon.  Modify or re-create that
configuration. This is often done via a conversion script in
/arch/RF/llrf/ASCIIConfig. If you used another method to create the
engine configuration, this is a good time to remember what you did.

Then, to actually use that new config file, the engine needs to
restart. We could simply wait for the next scheduled restart, in our
example the next Wednesday, 10:20. Alternatively, we can run the
llrf/stop-engine.sh script.
Watch the RF daemon, for example via /arch/RF/view-daemon.sh.
Within a few minutes, it ought to detect that the engine had stopped
and then restart it.

\subsection{Add Engine or Daemon}
Edit archiveconfig.xml to define the new engine under an existing
demon. Or add a line for a new daemon, then add the new engine under
it.  Unless you feel lucky today, use ``xmllint -valid
archiveconfig.xml'' to assert that you preserved the basic
well-formedness of the XML document.

Invoke ``scripts/update\_archive\_tree.pl''. Per default, it will re-create all
daemon and engine directories, so you might want to use the ``-s''
option to limit its operation to the new or modified subsystem.

In case the daemon was already running, it won't learn about the new
engine unless you restart it. So run the ``stop-daemon.sh'' script
followed by ``run-daemon.sh'' in the daemon directory to restart
the daemon, which will then start any newly added engines.

At the SNS, the data server on web2 also needs a manually created
``/arch/daemon/engine/current\_index'' soft link for newly added
engines which points to the ``current\_index'' on the read-only NFS share.

\subsection{An Engine isn't running}
Check the process list to assert that the engine in question is really not
running (on UNIX, try ``ps -aux''). Look again. If the engine is
actually running but not responding via its HTTPD, check its CPU usage
and the dates and sizes of the files in the sub-archive that the
engine is supposed to write. Is it adding to the data files?
It might respond again after a few minutes, although
this of course indicates that your computer is overloaded and you have
to reevaluate how much you can archive on that machine.
If all else fails, remove the engine process.

Check the log file of the engine, generated in the engine
subdirectory, for any clues. If you are convinced that the engine is
not running, but find an ``archive\_active.lck'' lock file in the
engine directory, remove it. Now the daemon should be able to start
your engine.  

\subsection{I want to stop a Daemon}
Run stop-daemon.sh.

\subsection{A Daemon isn't running}
Run start-daemon.sh in the daemon directory. If the daemon keeps quitting,
check its log file for clues.

\subsection{\INDEX{Remove Channels, Data}}
As for removing data from within one sub-archive, see 
\ref{sec:deleteDataFromArchive}.
You can remove a whole sub-archive, for example a subdirectory
``daemon/engine/2006/01\_10'', by deleting that directory and then rebuilding
all affected indices.

\subsection{\INDEX{Re-build Indices}}
Whenever you add or remove a sub-archive, the indices for that engine
and daemon (indexconfig.xml, maybe also binary master\_index)
become obsolete: They might still list data in a
sub-archive that you removed, or not yet include a new
sub-archive.

In case you know that the only change is \emph{added} data, a run of the
update\_indices.pl script should suffice. But whenever data has been
removed, rearranged, or you suspect a \INDEX{broken master index},
because you can retrieve data from the individual sub-archives but not
via the master index, the recipe is as follows:
\begin{itemize}
\item Delete the indexconfig.xml and master\_index files that refer to
      the data. For example, when you rearranged data in RF/llrf/2006,
      you need to delete the indexconfig.xml and master\_index files
      in RF/llrf and RF.
\item Invoke update\_indices.pl, or wait until update\_server.pl
      does this for you, triggered by a cron job.
\end{itemize}

\subsection{More Data Management}
See the description of the ArchiveDataTool in section \ref{sec:datatool}.
