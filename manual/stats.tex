\section{Statistics}
It is impossible to provide universal performance numbers for the
components of the ChannelArchiver toolset. Tests of a realistic setup
are always influenced by network delays: IOCs communicate with ArchiveEngines,
data client tools query data from the network data server.
And while the archiver tools of course share the CPU with all the
other applications that happen to run on the same CPU,
the CPU speed is less important. Most crucial is probably the hard
disk performance. Access to data on NFS-mounted disks is by orders of
magnitude slower than access to data on local disks.

The following are performance values obtained on a computer
with an 1~GHz CPU, an ordinary IDE disk, that was mostly idle
while the archiver tools ran. % That's blitz.ta53.lanl.gov
The corresponding values on a machine with an 800~MHz CPU, concurrently
used by other people, but faster hard disks (Mylex
DAC960PTL1 PCI RAID Controller with 5 Quantum Atlas 10K drives) were
slightly better.

We also provide some comparison to the previous architecture that used
the same data file format but instead of the RTree-based index there
were ``Directory Files''. Instead of being able to combine several sub-archives
into one index, one utilized an ASCII ``Master File'' that simply listed the
sub-archives.

\subsection{Write Performance}
As a baseline for raw data writing speed, the 'bench' program that can
be found in the ChannelArchiver/Engine directory consistently writes
at least 80000 values per second on the test computer.

\subsection{Index Performance}
Performance and index size depends on the $M$ value configuration of
the RTree.
\begin{itemize}
% LANL Xmtr Data 2002:
\item 12 sub-archives, 1.2~MB of old directory files, 1.4~GB Data
      files:\\
      Converting directory files into index files with $M=50$:
      3~minutes, resulting in 11~MB for the new index files.\\
      Creating a master index: 35~seconds for a master index of 9~MB.\\
      Re-run of the ArchiveMegaindex tool: 7~seconds.
      This shows that the ArchiveIndexTool detects when data blocks
      are already in the master index and does not add them again.
% LANL Xmtr Data 2003:
\item 92 sub-archives, 12~MB directory files, 2.3GB of data files:\\
      Converting into 61~MB of index files: About 12 minutes.\\
      Creating a master index: about 2~minutes, the resulting
      index uses about 1.1~GB.
      (Values for $M=10$: Individual indices sum to 25~MB. Master
      index took 60~minutes and was 1.4~GB.)\\
      Re-run: 40~minutes. Again the master index does not change at
      all because the ArchiveIndexTool recognizes that the data blocks
      from the sub-archives have already been added. But finding the
      last data block of the about 500 channels from the 92
      sub-archives in the 1.1~GB master index file cannot be cached by
      the operating system as it could in the previous case with a
      24~MB master index.
\end{itemize}

\noindent With the 92-subarchive index, the following was tested:
\begin{itemize}
\item Time to list all channel names: $<$1~second.\\
      This took about 7 seconds with the old ``multi archive'' file
      that required opening the individual sub-archives.
\item Time to find channels that match a pattern and show their
      start/end times: 0.2~seconds (4 channels out of 500 matched).\\
      This took about 6~seconds with the previous ``multi archive''.
\item Seek test, i.e.\ find a data block for a given start time:
      Index file requires $<$0.5~seconds, old index uses 1.5~seconds.
\end{itemize}

\subsection{Retrieval Performance}
Tests of the retrieval performance often include not only the
code for getting at the data but also for presenting it. In the
case of the command-line ArchiveExport program this means the process
of converting time stamps and values into ASCII text and printing them.
The output was redirected to /dev/null to avoid additional penalties.

\begin{itemize}
\item Dump all the 143000 values for a channel: 4~seconds,
      translating into 35700 values per second.
\end{itemize}
