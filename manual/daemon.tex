\chapter{ArchiveDaemon} \label{ch:daemon}
\begin{figure}[htb]
\begin{center}
\InsertImage{width=0.8\textwidth}{daemon}
\end{center}
\caption{\label{fig:daemon}Archive Daemon, refer to text.}
\end{figure}

\noindent The ArchiveDaemon is a script that automatically starts,
monitors and restarts ArchiveEngines on the local host. It includes a
built-in web server, so by listing all the ArchiveEngines that are
meant to run on a host in the ArchiveDaemon's configuration file, one
can check the status of all these engines on a single web page as
shown in Fig.~\ref{fig:daemon}.

Chapter \ref{ch:examplesetup}, ``Example Setup'', on page
\pageref{ch:examplesetup}, gives more details on the suggested use of
the ArchiveDaemon. The daemon will attempt to start any ArchiveEngine
that it does not find running. In addition, the daemon can
periodically stop and restart ArchiveEngines in order to create
e.g.\ daily sub-archives.  Furthermore, it adds information about
each sub-archive of newly created ArchiveEngines to a mailbox directory
so that the index mechanism can create the necessary indices
and update the data server configuration.

Before using the ArchiveDaemon, one should be familiar
with the configuration of the ArchiveEngine (sec.\ \ref{sec:engine}),
and how to start and stop it. Furthermore, one needs to be familiar
with the ArchiveIndexTool (sec.\ \ref{sec:indextool}).

\subsection{Configuration}
The ArchiveDaemon expects to find a configuration file called
``ArchiveDaemon.xml'' in the directory where it is started.  That
configuration file, an example of which can be found in listing
\ref{lst:daemonconfigex}, needs to follow the DTD from listing
\ref{lst:daemonconfigdtd}.

In many cases you might not configure the daemon yourself,
but instead use the mechanism described in chapter \ref{ch:examplesetup}.

\lstinputlisting[float=htb,keywordstyle=\sffamily,caption={Example Archive Daemon Configuration},label=lst:daemonconfigex]{../ExampleSetup/ArchiveDaemon.xml}

\lstinputlisting[float=htb,keywordstyle=\sffamily,caption={XML DTD for
    the Archive Daemon Configuration},label=lst:daemonconfigdtd]{../ExampleSetup/ArchiveDaemon.dtd}

\noindent The configuration lists all the
ArchiveEngines that the daemon should manage on the local
computer. One ``engine'' element per ArchiveEngine specifies the
configuration of each engines. Specifically, the following tags are
allowed:

\subsubsection{desc}
This mandatory element is used for the ``-description'' option of the
Archive Engine, see section \ref{sec:enginedesc}.

\subsubsection{port}
This mandatory element determines the port number of the engine's HTTP
server, see section \ref{sec:engineport}.

\NOTE The ArchiveDaemon itself requires a TCP port number for its HTTP
server. The port numbers used by the ArchiveDaemon and all the Archive
Engines need to be different. You cannot use the same port number more
than once per computer.

\subsubsection{config}
This mandatory element must contain the path to the configuration
file of the respective ArchiveEngine, see section \ref{sec:engineconfig}.
This can be either the full path to the engine configuration file or a path
beneath the current working directory in which the ArchiveDaemon is running.

\subsubsection{daily}
This optional element configures the ArchiveDaemon to restart the
ArchiveEngine periodically. The element must contain a time in the
format ``HH:MM'' with 24-hour hours HH and minutes MM. One example
would be ``02:00'' for a restart at 2~am each morning.

\subsubsection{weekly}
Weekly is similar to daily, but using an element that contains the day
of the week (Mo, Tu,  We, Th, Fr, Sa, Su) in addition to the time
on that day in 24-hour format, e.g.\ ``We 08:00''. In this example,
the daemon will attempt a restart every Wednesday, 8'o clock in the morning.

\subsubsection{timed}
In this case, the element needs to contain a start/duration time pair
in the format ``HH:MM/HH:MM''. The first, pre-slash 24-hour time stamp
indicates the start time, and the second 24-hour time, trailing the
slash, specifies the runtime. The engine will be launched at the
requested start time and run for the duration of the runtime. As an
example, ``08:00/01:00'' requests that the daemon starts the engine at
08:00 and stops it after one hour, probably around 09:00.

\subsubsection{hourly}
This optional element configures the ArchiveDaemon to restart the
ArchiveEngine periodically. The element must contain a number
specifying hours: A value of 2.0 will cause a restart every 2
hours. The hourly restart is quite inefficient and primarily meant for testing.

\subsection{Starting and Running}
The ArchiveDaemon is a perl script that is typically started like this:

\begin{lstlisting}[keywordstyle=\sffamily]
$ cd whereever_you_placed_ArchiveDaemon.xml
$ perl ArchiveDaemon.pl -f ArchiveDaemon.xml
Read ArchiveDaemon.xml, will disassociate from terminal
and from now on only respond via
          http://localhost:4610
You can also monitor the log file:
          ArchiveDaemon.log
\end{lstlisting}

\noindent One can use any web browser to connect to the daemon's HTTP server
under the URL shown in the above status message. Fig.~\ref{fig:daemon}
shows one example. The ArchiveDaemon offers a command line option for
selecting a specific TCP port.
Whenever running more than one ArchiveDaemon per computer, they
need to be started with different TCP port numbers. Furthermore, each
ArchiveEngine needs a different TCP port number.
\begin{lstlisting}[keywordstyle=\sffamily]
USAGE: ArchiveDaemon [options] 
Options:
  -p <port>    : TCP port number for HTTPD
  -f <file>    : config. file
  -i <URL>     : path or URL to indexconfig.dtd
  -u <minutes> : Update period for master index
  -d           : debug mode (stay in foreground etc.)
\end{lstlisting}

\noindent The first few lines of the ArchiveDaemon.pl script contain
numerous configuration variables. They allow fine tuning of e.g.\ how
often the daemon queries the Archive Engines and other customization
options. In there you can also change many of the file names from
their defaults up to the point where none of the following
applies. With the original settings, the ArchiveDaemon will create or
use the following files in the directory in which it was started:
\begin{itemize}
\item ArchiveDaemon.log\\
  The log file of the ArchiveDaemon.
\item indexconfig.xml\\
  This is a configuration file for the ArchiveIndexTool.  If the file
  already exists, the ArchiveDaemon will add every new sub-archive
  that it creates to the file. If the file does not exist, one will be
  created the next time a new sub-archive is started.

  Note that the ArchiveDaemon will \emph{not} search for sub archives
  or index files and automatically add them to indexconfig.xml. It
  will only add newly created sub-archives. If you already have
  sub-archives that need to be included in the master index, your
  initial indexconfig.xml needs to list them.
  Also note that the ArchiveDaemon reads this file on startup. In
  order to \emph{remove} indices from indexconfig.xml, it is therefore
  required to stop the running daemon, since it will otherwise
  overwrite indexconfig.xml with its in-memory version.
\item indexupdate.xml\\
  This file is similar to indexconfig.xml, except that the index files
  for sub-archives that are older than the master index file are
  commented out, assuming that the master index already contains all
  the information from those older sub-archives.
  Running ArchiveIndexTool with this indexupdate file is naturally faster
  than using indexconfig.xml. The ArchiveDaemon uses the full index
  configuration from indexconfig.xml once after startup, and then
  switches to indexupdate.xml.
\item ArchiveIndexTool.log\\
  The log file of the last run of the ArchiveIndexTool.
\item master\_index\\
  The ArchiveIndexTool is run with indexconfig.xml to update
  this master index file.
\end{itemize}

\noindent The ArchiveDaemon configuration file must list the full path
names to the configuration files for the ArchiveEngines or use a path
that is below the current working directory of the ArchiveDaemon.
Within each of those directories, an ArchiveEngine is run and the
following files will be created:
\begin{itemize}
\item ArchiveEngine.log\\
  A log file for the ArchiveEngine running in that directory
\item archive\_active.lck\\
  Lock file of the ArchiveEngine
\item YYYY/MM\_DD/index \\
  A subdirectory for index and data files of the sub-archive.  If the
  ArchiveDaemon is configured to perform daily restarts, the format
  uses the year, month and day to build the path name.
\end{itemize}

\subsection{Daemon's Web Pages}
The main web page of the ArchiveDaemon's HTTPD looks similar to
Fig.~\ref{fig:daemon}. You can use any web browser to look at the
daemon's web pages. The URL follows the format
\begin{lstlisting}[keywordstyle=\sffamily]
    http://host:port
\end{lstlisting}
\noindent where ``host'' is the name of the computer where the
ArchiveDaemon is running. More often than not you will use
``localhost''. ``Port'' is the TCP port that was specified as a
command-line option to the ArchiveDaemon program, otherwise it
defaults to 4610. So the default URL will be http://localhost:4610.

The main page lists all the archive engines that this daemon controls
with their status. The first column also contains links to the
individual archive engines. The status shows any of the following:

\begin{itemize}
\item ``N/M channels connected''\\
      This means the ArchiveEngine is running and responding,
      telling us that N out of a total of M channels have connected.
      If not all channels could connect, you might want to follow
      the link to the individual engine to determine what channels are
      missing and why: Is an IOC down on purpose? Is an IOC
      disconnected because of network problems? Does a channel simply
      not exist, i.e.\ the engine's configuration is wrong?
\item ``Not Runnnig''\\
      This means that the respective ArchiveEngine did not respond
      when we queried it, and there is no ``archive\_active.lck'' lock
      file. This combination usually means that the engine is really
      not running (except for the Note below).

      The first step in debugging would be to check the engine's
      directory for a log file. Does it indicate why the engine could
      not start? Then check the daemon's log file. It should list the
      exact command used to start the engine. You can try that
      manually to check why it didn't work.
\item ``Unknown. Found lock file''\\
      This means that the respective ArchiveEngine did not respond
      when we queried it, but there is an ``archive\_active.lck'' lock
      file. This could have two reasons. It could mean that the engine
      is running but it was temporarily unable to respond to the
      daemon's request. An example would be that the engine is really
      busy writing and dealing with ChannelAccess, so that its web
      server had to wait and the daemon timed out. All should be fine
      again after some time.

      If, on the other hand, the situation persists, it usually means
      that the engine is hung or has crashed, so that it does not
      respond and the lock file was left behind.
      See Crashes on page \pageref{sec:crash}.
\end{itemize}

\NOTE The daemon queries only every once in a while and leave the
engines alone for most of the time.
Especially after startup, all engines will show up as ``Not Running''
in the daemon's web page while in fact most of them are already
running. Then you will see many disconnected channels while the
engines did in fact already connect to all channels. 
If you are impatient, you can click on the links to the individual
engines to get a more up-to-date snapshot of each engine's status.

\subsection{Disabling Engines}
The web interface of the daemon contains a link for each engine that
disables the engine. This places a file ``\INDEX{DISABLED.txt}'' in
the engine directory and stops the engine.
The daemon will not attempt to start engines as long as the ``DISABLED''
file is found. This is a convenient way for temporarily disabling
engines without removing them from the daemon's configuration.

\subsection{Stopping and More}
To stop the ArchiveDaemon, access the ``/stop'' URL of the daemon's
HTTPD, e.g. ``http://localhost:4610/stop''.  Similar to the
ArchiveEngine's HTTPD, this URL is not accessible by following links
on the HTTPD's web pages. You will have to type the URL. This is to
prevent web robots or a monkey who is sitting in front of the computer
and clicking on every link from accidentally stopping the daemon.
Finally, the daemon will respond to the URL ``/postal'' by stopping every
ArchiveEngine controlled by the daemon, followed by stopping itself.
