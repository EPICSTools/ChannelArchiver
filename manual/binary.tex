\section{Data Files}
The data files store the actual data, that is the time stamps, values
and the meta information like display limits, alarm limits and
engineering units. 
The archiver stores data for many channels in the same data
file. There aren't separate data files per channel because that would
produce too many files and slow the archiver down.  The names of the
data files look like time stamps. They are somewhat related to the
time stamps of the samples in there: The name reflects when the data
file was created. We then continue to add samples until the engine
decides to create a new data file. This means that a data file with a
name similar to yesterday's date can still be filled today.

\noindent{\bfseries Conclusion 1:} Ignore the names of the data files,
they don't tell you anything of use about the time range of samples
inside.

\subsection{Implementation Details}

\begin{table}[htbp]
  \begin{center}
    \begin{tabular}{ll}
     Offset  & Content \\
     \hline
     ...     & ... \\
     0x1000  & \underline{Numeric CtrlInfo} \\
             & display limits, units, ... \\
     ...     & ... \\
     0x2000  & \underline{Data Header} \\
             & prev buffer: ````, 0 \\
             & next buffer: ``X``, 0x4000 \\
             & CtrlInfo: 0x1000 \\ 
             & dbr\_type: dbr\_time\_double \\
             & buffer size, amount used, ... \\
             & \underline{Buffer:} dbr\_time\_double, dbr\_time\_double, ... \\
     ...     & ... \\
     0x4000  & \underline{Data Header} \\
             & prev buffer: ``X``, 0x2000 \\
             & next buffer: ``Y``, 0x4000 \\
             & CtrlInfo: 0x1000 \\ 
             & ... \\
             & \underline{Buffer:} dbr\_time\_double, dbr\_time\_double, ... \\
     ...     & ... \\  
    \end{tabular}
    \caption{Data file: Example layout for a data file ``X''}
    \label{tab:datafile}
  \end{center}
\end{table}

\noindent Table~\ref{tab:datafile} shows the basic layout of a data file ``X''.
Most important, the data file only stores data. It doesn't know about
the channel names to which the data belongs. The index would for
example tell us that the data of interest for channel ``fred'' can be
found in data file ``X'' at offset 0x2000. In there, the Data Header
points to the preceding buffer (none in this case) and the following
buffer (in this case: same file, offset 0x4000). It also provides the
data type, size and number of samples to be found in the Data Buffer
which immediately follows the Data Header. 

\noindent{\bfseries Conclusion 2:} A data file is useless without the
accompanying index file.

\noindent The Data Buffer contains the raw dbr\_time\_xxx-type values
as received from ChannelAccess. The meta information, that is: limits,
engineering units or for enumerated channels the enumeration strings,
are stored in a CtrlInfo block. Each Data Header contains a link to a
CtrlInfo block, in this case one at offset 0x1000 which happens to
contain numeric control information.
Each buffer contains a certain number of samples. Whenever a buffer is
full, a new one is added. The new buffer might be created at the end
of the same data file, but the engine might also create a new data
file after a certain time or whenever a data file gets too big.
In the example from Table~\ref{tab:datafile}, the first buffer at offset
0x2000 links to a next buffer at offset 0x4000 in the same file ``X'',
and that buffer in turn points to another buffer in a different file
``Y''. Note that both the buffer at offset 0x2000 and the one at
offset 0x4000 share the same meta information at offset 0x1000,
probably because the meta information has not changed.

\noindent{\bfseries Conclusion 3:} Do not delete individual data
files, because this will break the links between data files and result
lost samples. Do not remove the index file. All the data files that were
created in one directory together with an index file need to stay together.
You can move the index and all data files into a different directory, but
you must not remove or rename any single data file.


