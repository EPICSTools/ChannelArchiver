\section{What is a Channel?} % -----------------------------------------------
The Channel Archiver deals with Channels that are served by EPICS
ChannelAccess. It stores all the information available via ChannelAccess:
\begin{itemize}
\item Time Stamp
\item Status/Severity
\item Value
\item Meta information:\\
      Units, Limits, ... for numeric channels,
      enumeration strings for enumerated channels.
\end{itemize}

\noindent The archiver stores the original \INDEX{time stamps} as it receives
them from ChannelAccess. It cannot check if these time stamps are valid, except
that it refuses to go ``back in time'' because it can only append new
values to the end of the data storage. It is therefore imperative to
properly configure the data sources, that is: the clocks on the CA
servers. For more details on the EPICS time stamps refer to section
\ref{sec:GMT}.

\label{back:in:time}
\NOTE If the CA server provides bad time stamps, for example stamps
that are older than values which are already in the archive, or stamps
that are unbelievably far ahead in the future, the ArchiveEngine will log
a warning message and refuse to store the affected samples.
This is a common reason for ``Why is there no data in my archive?''.
(There is one more, hard to resolve reason for back-in-time warnings, see
page \pageref{sec:back-in-timefaq}).

As for the values themselves, the native data type of the channel as
reported by ChannelAccess is stored. For those familiar with the
ChannelAccess API, this means:
Channels that report a native data type of DBR\_xxx\_ are stored as
DBR\_TIME\_xxx after once requesting the full DBR\_CTRL\_xxx information.
 The Archiver can therefore handle scalar and array numerics
(double, int, ...), strings and enumerated types. 

\section{Data Sources} \label{sec:datasource} % ------------------------------
Before even considering the available \INDEX{sampling options}, it is
important to understand the \INDEX{data sources}, the \INDEX{ChannelAccess
servers} whose channels we intend to archive.
In most cases we will archive channels served by an EPICS Input/Output
Controller (\INDEX{IOC}) which is configured via a collection of EPICS
\INDEX{records}.
Alternatively, we can archive channels served by a custom-designed CA
server that utilizes the portable CA library \INDEX{PCAS}.
In those cases, one will have to contact the implementor of the custom
CA server for details.
In the following, we concentrate on the IOC scenario and use the
analog input record from listing~\ref{lst:airecord} as an example.

\lstinputlisting[float=htb,caption={``aiExample'' record},label=lst:airecord]{aiexample.db} 

\noindent What happens when we try to archive the channel ``aiExample''?
We will receive updates for the record's value field (VAL). In fact we
might as well have configured the archiver to use ``aiExample.VAL''
with exactly the same result.
The record is scanned at 10~Hz, so we can expect 10 values per second.
Almost: The archive dead band (ADEL) limits the values that we receive
via CA to changes beyond 0.1. When archiving this channel, we could
store at most 10 values per second or try to capture every change,
utilizing the ADEL configuration to limit the network traffic.

\NOTE The archiver has no knowledge of the scan rate nor the dead band
configuration of your data source! You have to consult the IOC
database or PCAS-based code to obtain these.

With each value, the archiver stores the time stamp as well as the
status and severity. For aiExample, we configured a high limit of 10
with a MAJOR severity. Consequently we will see a status/severity of
HIHI/MAJOR whenever the VAL field reaches the HIHI limit.
In addition to the value (VAL field), the archiver also stores certain
pieces of \INDEX{meta information}. For numeric channels, it will store the
engineering units, suggested display precision, as well as limits for
display, control, warnings, and alarms. For enumerated channels, it
stores the enumeration strings.
Applied to the aiExample record, the suggested display precision is
read from the PREC field, the limits are derived from HOPR, LOPR,
HIHI, ..., LOLO.

\NOTE You will have to consult the record reference manual or even
record source code to obtain the relations between record fields and
channel properties. The analog input record's EGU field for example
provides the engineering units for the VAL field. We could, however,
also try to archive aiExample.SCAN, that is the SCAN field of the same
record. That channel aiExample.SCAN will be an \emph{enumerated} type
with possible values ``Passive'', ``.1 second'' and so on. The EGU
field of the record no longer applies!
Another example worth considering: While HOPR defines the upper
control limit for the VAL field, what is the upper control limit if we
archive the HOPR field itself?

It is also important to remember that the archiver
--- just like any other ChannelAccess client --- does {\bfseries not} know
anything about the underlying EPICS record type of a channel. In fact
the channel might not be based on any record at all if we use a
PCAS-based server.
Given the name of an analog input record, it will store the record's
value, units and limits, that is: most of the essential record
information. Given the name of a stepper motor record, the
archiver will also store the record's value (motor position) with the
units and limits of the motor position. It will not store the
acceleration, maximum speed or other details that you might consider
essential parts of the record. To archive those, one would have to
archive them as individual channels.

