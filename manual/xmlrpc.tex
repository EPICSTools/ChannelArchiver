\section{\INDEX{XML-RPC Protocol}} \label{sec:xmlprotocol}
The following is a description of the calls implemented by the archive
data server based on the XML-RPC protocol.
For details on XML-RPC,  including the specifications and examples of
how to use it from within C,  C++, Java, perl, please refer to
\href{http://www.xmlrpc.com}{http://www.xmlrpc.com}.

Users of Java should probably utilize the Java archive data client
library provided with the ChannelArchiver. Users of other programming
environments need to refer to the following.

\subsection{archiver.info} \label{sec:archiver:info} % ------------------- 
This call returns version information. It will allow future
compatibility if clients check for the correct version numbers.  In
addition, it provides hints on how to decode the values served by this
server.

\begin{lstlisting}[keywordstyle=\sffamily]
{ int32             ver,
  string            desc,
  string            how[],
  string            stat[],
  { int32 num,
    string sevr,
    bool has_value,
    bool txt_stat
  }                 sevr[]
 } = archiver.info()
\end{lstlisting}

\begin{description}
\item[\sffamily ver:]  Version number. Described in here is version '1'.
\item[\sffamily desc:] Cute description that one can print.
\item[\sffamily how:]  Array of strings with a description of the
                       request methods supported for 'how' in the
           		       archiver.values() call described further below in
                       section \ref{sec:archiver:values}.
\item[\sffamily stat:] Array of strings with a description of the
                       ``status'' part of the values returned by
                       the archiver.values() call.     
\item[\sffamily sevr:] Array of structures with a description of the
                       ``severity'' part of the values returned by
                       the archiver.values() call.     
\end{description}

\noindent The result is a structure with a numeric ``ver''
member, a string ``desc'' member and so on as listed above.
Implementations like perl will return a hash with members
``ver'', ``desc'', etc.
The strings in ``how'' describe the request method for how=0, how=1,
and so on.
The strings in ``stat'' describe the enumerated status values, the
typical result is shown in table \ref{tab:stat}.

The more important information is in the ``sevr'' array.
It also lists severity numbers (``num'') and their associated string
representation (``sevr''). In addition to the alarm severities defined
by the EPICS base software, the archiver uses some special severity
values which have the ``has\_value'' property set to false. They
identify situations that have no value because the channel
was disconnected or the archiver was turned off. Other special
severities identify repeat counts which are used in the periodic
scanning modes of the archive engine: If the channel did not change
for N sample times, a repeat count of N is logged instead of logging
the same value N times. In that case, the ``txt\_stat'' property is
set to false because the status (stat) field no longer corresponds to
a status string from table \ref{tab:stat}. Instead, it provides the
repeat count N.
Table \ref{tab:sevr} lists the typical content of the ``sevr'' array,
table \ref{tab:statsevrexample} presents examples for decoding values
based on their status and severity information.

\begin{table}[htbp]
  \begin{center}
    \sffamily
    \begin{tabular}[t]{l|l}
    Array Element & String \\
    \hline
      0   & NO\_ALARM   \\
      1   & READ ALARM   \\
      2   & WRITE ALARM \\
      3   & HIHI ALARM  \\
      4   & HIGH ALARM  \\
      5   & LOLO ALARM  \\
      6   & LOW ALARM  \\
      7   & STATE ALARM  \\
      \ldots   & \ldots  \\
      17   & UDF ALARM  \\
      \ldots   & \ldots  \\
    \end{tabular}
    \caption{Alarm Status Values returned in the ``stat'' member of archiver.info()}
    \label{tab:stat}
  \end{center}
\end{table}

\begin{table}[htbp]
  \begin{center}
    \sffamily
    \begin{tabular}[t]{l|l|l|l}
     num & sevr             & has\_value & txt\_stat \\
    \hline
       0 & NO\_ALARM        & true       & true  \\
       1 & MINOR            & true       & true  \\
       2 & MAJOR            & true       & true  \\
       3 & INVALID          & true       & true  \\
    3968 & Est\_Repeat      & true       & false \\
    3856 & Repeat           & true       & false \\
    3904 & Disconnected     & false      & true  \\
    3872 & Archive\_Off     & false      & true  \\
    3848 & Archive\_Disabled& false      & true
    \end{tabular}
    \caption{Alarm Severity Values returned in the ``sevr'' member of archiver.info()}
    \label{tab:sevr}
  \end{center}
\end{table}

\begin{table}[htbp]
  \begin{center}
    \sffamily
    \begin{tabular}[t]{l|l|l|l}
    Severity (sevr) & Status (stat) & Value & Example Text \\
    \hline
                  0 &            0  &  3.14 & ``3.14'' \\
                  1 &            6  &  3.14 & ``3.14 MINOR LOW'' \\
               3856 &            6  &  3.14 & ``3.14 Repeat 6'' \\
               3904 &            0  &  0    & ``Disconnected'' \\
    \end{tabular}
    \caption{Examples for decoding samples returned from the
    archiver.values() call based on their Status and Severity}
    \label{tab:statsevrexample}
  \end{center}
\end{table}

\subsection{archiver.archives} % --------------------------------------------
Returns the archives that this data server can access.

\begin{lstlisting}[keywordstyle=\sffamily]
{ int32 key, 
  string name, 
  string path }[] = archiver.archives()
\end{lstlisting}

\begin{description}
\item[\sffamily key:] A numeric key that is used by the following
                      routines to select the archive.
\item[\sffamily name:] A description of the archive that one could
                       e.g.\ use in a drop-down selector in a GUI
                       application for allowing the user to select an archive.
\item[\sffamily path:] The path to the index file,  valid on the file
                       system where the data server runs.
                       It might be meaningful to a few users who want to
                       know exactly where the data resides,  but it is
                       seldom essential for XML-RPC clients to look at this.
\end{description}

\noindent The result is an array of structures with a numeric ``key''
member and strings ``name'' and ``path''.
An example result could be:
\begin{lstlisting}[keywordstyle=\sffamily]
{ key=1,  name="Vacuum", path="/home/data/vac/index" },
{ key=2,  name="RF",     path="/home/data/RF/index" }
\end{lstlisting}

\noindent So in the following one would then use key=1 to access
vacuum data etc. One can expect the keys to be small,  positive
numbers,  but they are not guaranteed to be consecutive as 1, 2, 3,
... Since the keys could be something like 10,  20, 30 or 1, 17, 42,
they are not useful as array indices.

\subsection{archiver.names} % ------------------------------------------
Returns channel names and start/end times.
The key must be a valid key obtained from archiver.keys.
Pattern is a \INDEX{regular expression};
if left empty,  all names are returned.

\NOTE The Time Stamps are \emph{not} the raw EPICS time stamps with 1990 epoch, 
but use the time\_t data type based on a 1970 epoch.

\begin{lstlisting}[keywordstyle=\sffamily]
{string name, 
 int32 start_sec,  int32 start_nano,
 int32 end_sec,    int32 end_nano}[] 
         = archiver.names(int32 key,  string pattern)
\end{lstlisting}

\noindent The result is an array of structures,  one structure
per channel that matches the pattern.
Start/end gives an idea of the time range that can
be found in the archive for that channel.
The archive might actually contain entries \emph{after}
the reported end time because the index might not
be up too date on the end times.

\subsection{archiver.values} \label{sec:archiver:values} % -----------------
This call returns values from the archive identified by the key for a
given list of channel names and a common time range.

\begin{lstlisting}[keywordstyle=\sffamily]
result = archiver.values(
         int key, 
         string name[], 
         int32 start_sec,  int32 start_nano,
         int32 end_sec,  int32 end_nano, int32 count,
         int32 how)
\end{lstlisting}

\noindent The parameter ''how'' determines how the raw values of the
various channels get arranged to meet the requested time range and
count.  For details on the methods mentioned in here refer to section
\ref{sec:timestampcorr} and following, beginning on page
\pageref{sec:timestampcorr}.
\begin{description}
\item[\sffamily how = 0 (raw):]
  Get raw data from archive (see \ref{sec:rawdata}), starting w/ 'start',
  up to either 'end' time or max. 'count' samples.
\item[\sffamily how = 1 (spreadsheet):]
  Get data that is filled or staircase-interpolated, starting
  w/ 'start', up to either 'end' time or max. 'count' samples
  (see \ref{sec:filling}).
  For each channel, the same number of values is returned. The
  time stamps of the samples match accross channels, so that one can
  print the samples for each channel as columns in a spreadsheet.
  If a spreadsheet cell is empty because the channel does not have any
  useful value for that point in time, a status/severity of
  UDF/INVALID is returned (Tables \ref{tab:sevr} and \ref{tab:stat}).
\item[\sffamily how = 2 (averaged with 'count'):]
  Get averaged data from the archive, starting w/ 'start',
  up to either 'end' time or max. 'count' samples
  (see \ref{sec:lininterpol}).
  The data is averaged within bins whose size is determined by 
  of (end-start)/count, so you should expect to get close to 'count'
  values which cover 'start' to 'end'.
\item[\sffamily how = 3 (plot binning):]
  Uses the plot-binning method based on 'count' bins
  (see \ref{sec:plotbinning}).
  For new plot tools, the averaged method (how=5) should be considered,
  since it automatically falls back to 'raw' data.
\item[\sffamily how = 4 (linear):]
  Get linearly interpolated data from the archive, starting w/ 'start',
  up to either 'end' time or max. 'count' samples
  (see \ref{sec:lininterpol}).
  The data is interpolated onto time slots which are multiples
  of (end-start)/count, so you should expect to get close to 'count'
  values which cover 'start' to 'end'.
\item[\sffamily how = 5 (averaged):]
  Get averaged data from the archive, starting w/ 'start',
  up to the 'end' time (see \ref{sec:lininterpol}).
  The data is averaged within bins whose size is determined by 
  the 'count'. A value of 60 means: Provide an averaged value every 60 seconds,
  together with the minimum and maximum.
  
  If there is only a single value within an averaging window, or a value
  that cannot be averaged because it is a string or a special info value,
  that single value is returned.
  The idea is that we get down to raw values when zooming in far enough to
  get small bin sizes.
\end{description}
Refer to section \ref{sec:timestampcorr} for example results.

\noindent The result is an array of structures,  one structure per
requested channel:

\begin{lstlisting}[keywordstyle=\sffamily]
result := { string name,  meta, int32 type,
            int32 count,  values }[]
\end{lstlisting}

\begin{description}
\item[\sffamily name:]
   The channel name.
   Result[i].name should match name[i] of the request, 
   so this is a waste of electrons,  but it's sure convenient
   to have the name in the result,  and we're talking XML-RPC,
   so forget about the electrons.
\item[\sffamily meta:]
   The meta information for the channel. This is itself a structure 
   with the following entries:
   \begin{lstlisting}[keywordstyle=\sffamily]
meta := { int32 type;
          type==0: string states[], 
          type==1: double disp_high, 
                   double disp_low, 
	           double alarm_high, 
                   double alarm_low, 
                   double warn_high, 
                   double warn_low, 
                   int prec,  string units
        }
   \end{lstlisting}
\item[\sffamily type:]
   Describes the data type of this channel's values:
  \begin{lstlisting}[frame=none, keywordstyle=\sffamily]
  string   0
  enum	   1 (XML int32)
  int      2
  double   3
  \end{lstlisting}
\item[\sffamily count:]
  Describes the array size of this channel's values,  using 1 for
  scalar values. Note that even scalar values are returned as an array
  with one element!
\item[\sffamily values:]
  This is an array where each entry is a structure of the following
  layout:
  \begin{lstlisting}[frame=none, keywordstyle=\sffamily]
  values := { int32 stat,  int32 sevr,
              int32 secs,  int32 nano,
              <type> value[] } []
  \end{lstlisting}
  
  If the request was for averaged data, raw samples that cannot be averaged
  are returned as shown before, while truly averaged samples are returned with
  a minimum and maximum, and the 'value' element contains the average
  over the samples in the averaging bin:
  \begin{lstlisting}[frame=none, keywordstyle=\sffamily]
  values := { int32 stat,  int32 sevr,
              int32 secs,  int32 nano,
              double value[],
              double min, double max } []
  \end{lstlisting}
\end{description}

\noindent The values for status and severity match in part those that
the EPICS IOC databases use. The ArchiveEngine simply receives and
stores them, they are passed on to the retrieval tools without
change. In addition, the archiver toolset uses special severity values
to indicate a disconnected channel or the fact that the ArchiveEngine
was shut down.  For details refer to section \ref{sec:archiver:info}
and the tables \ref{tab:stat}, \ref{tab:sevr} and
\ref{tab:statsevrexample}.

\subsection{Note about Tiny Numbers and Precision} \label{sec:xml:tiny} % -----------------
Some systems deal with small numers. Vacuum readings often use numbers
like $5 \times 10^{-8}$. The XML-RPC specification is unfortunately
unclear as to how numbers should get serialized and parsed other than
specifically prohibiting the exponential notation. The best one could
serialize the example number would therefore be ``0.00000005''.

When the ArchiveDataServer is build with the XML-RPC library for C/C++
as described in the installation section, \ref{sec:install}, it will
attempt to properly serialize small numbers.  When using another
XML-RPC library, and this includes the XML-RPC library that your
client program uses, small numbers might end up being serialized as
zero. The Java Archive Client appears to handle small numbers, as does
the ``Frontier'' XML-RPC library for perl.

A similar issue applies to the precision of floating point numbers:
The ArchiveDataServer serializes numbers with a fixed precision that
is determined by the XML-RPC library for C/C++. You can patch the
library to increase the precision.

